\documentclass[18pt,a4paper]{article}
\usepackage[utf8]{inputenc}
\usepackage{amsmath}
\usepackage{amsfonts}
\usepackage{amssymb}
\usepackage{listings}
\usepackage{color}
\usepackage{textcomp}
\definecolor{listinggray}{gray}{0.9}
\definecolor{lbcolor}{rgb}{0.9,0.9,0.9}
\usepackage{tikz}
\newcommand*\circled[1]{\tikz[baseline=(char.base)]{
            \node[shape=circle,draw,inner sep=1pt] (char) {#1};}}
%\author{kyriakosschwarz}
\title{Diskrete Mathematik 2}



%Math
\usepackage{amsmath}
\usepackage{amsfonts}
\usepackage{amssymb}
\usepackage{amsthm}
\usepackage{ulem}

%PageStyle
%\usepackage[ngerman]{babel} % deutsche Silbentrennung
\usepackage[utf8]{inputenc} 
\usepackage{fancyhdr, graphicx}
\usepackage[scaled=0.92]{helvet}
\usepackage{enumitem}
\usepackage{parskip}
\usepackage[a4paper,top=2cm]{geometry}
\setlength{\textwidth}{17cm}
\setlength{\oddsidemargin}{-0.5cm}


% Shortcommands
\newcommand{\Bold}[1]{\textbf{#1}} %Boldface
\newcommand{\Kursiv}[1]{\textit{#1}} %Italic
\newcommand{\T}[1]{\text{#1}} %Textmode
\newcommand{\Nicht}[1]{\T{\sout{$ #1 $}}} %Streicht Shit durch

%Arrows
\newcommand{\lra}{\leftrightarrow} 
\newcommand{\ra}{\rightarrow}
\newcommand{\la}{\leftarrow}
\newcommand{\lral}{\longleftrightarrow}
\newcommand{\ral}{\longrightarrow}
\newcommand{\lal}{\longleftarrow}
\newcommand{\Lra}{\Leftrightarrow}
\newcommand{\Ra}{\Rightarrow}
\newcommand{\La}{\Leftarrow}
\newcommand{\Lral}{\Longleftrightarrow}
\newcommand{\Ral}{\Longrightarrow}
\newcommand{\Lal}{\Longleftarrow}

%Mine(new)
\newcommand{\tab}{\hspace*{2em}}
\newcommand{\cmark}{\ding{51}}%
\newcommand{\xmark}{\ding{55}}%

\usepackage{amssymb}% http://ctan.org/pkg/amssymb
\usepackage{pifont}% http://ctan.org/pkg/pifont

%Metadata
%\fancyfoot[C]{If you use this documentation for a exam, you should offer a beer to the authors!}
\title{
	\vspace{5cm}
	Diskrete Mathematik 2 \\
}
%\author{Kyriakos Schwarz}
\date{FS 2013}





\begin{document}

% Titelbild
\maketitle
\thispagestyle{fancy}
\newpage

% Inhaltsverzeichnis
%\pagenumbering{Roman}
\tableofcontents	  	
\newpage

%\setcounter{page}{1}
\pagenumbering{arabic}

% Inhalt Start

\section{Erste Woche}


\subsection{Semesterablauf}

- Arithmetik in $\mathbb{Z}$\\
\\
- Modulares Rechnen\\
\\
- Gruppen\\
\\
- RSA\\
\\
- Quantifizierung\\
\\
- Induktion\\
\tab - Rekurision\\
\tab - Invarianten\\
\\
\rule{\textwidth}{0.4mm}\\
\\
- Kein Laptop\\
\\
- Zwischenpruefung: 30.04.2013 (1 Stunde)\\
\\
- 5. Maerz 2013 Unterricht nur bis 18:20\\
\\
\\
\rule{\textwidth}{0.4mm}\\
\\
- Buecher:\\
\\
\tab - Gries/Schneider\\
\tab \tab A logical approach to Discrete Math\\
\tab \tab Springer, 1993\\
\\
\tab - Jean Gallier\\
\tab \tab Discrete Math\\
\tab \tab Springer, 2010\\
\\
\tab - Struckermann/Waetiger\\
\tab \tab Mathematik fuer Informatiker\\
\tab \tab Spektrum, 2007\\
\\
\\
\rule{\textwidth}{0.4mm}\\
\\



\subsection{Quantifizierung}

$\mathbb{N} = \begin{cases} 
		 \{\underline{0},1,2,...\} \:(?)\\
		 \{1,2,...\} \:(?)\\ 
		 \end{cases}$ 


$\sum_{i=1}^{n}i^2 = 1^2 + 2^2 + ... + n^2 $\\
\\
$\sum_{i=1}^{-1}i^2 = \begin{cases}
                       ungueltig \:(?)\\
                       1^2 \:(?)\\
                       1^2 + 0^2 + (-1)^2 \:(?)\\
                       0 \:(\ra ja, \:Neutrales \:Element)\\
                      \end{cases}$\\                           
\\
$\sum_{i=1}^{n}i^2 + 1 = 1^2 + 2^2 + ... + n^2 + 1 $ (?)\\
\\
$\sum_{i=1}^{n}(i^2 + 1) = (1^2 + 1) + (2^2 + 1) + ... + (n^2 + 1) $ (?)\\
\\
$\sum_{\substack{i=1\\odd(i)}}^{n}i^2 = 1^2 + 3^2 + ... + n^2 $ , falls $odd(n)$, sonst $(n-1)^2$\\ 
\\
\rule{\textwidth}{0.4mm}\\
\\
$\prod_{i=1}^{n}i^2 = 1^2 * 2^2 * ... * n^2 $\\
\\
$\prod_{i=1}^{-1}i^2 = 1 $ (neutrales Element)\\
\\
\\
\rule{\textwidth}{0.4mm}\\
\\
(Java ==)\\
\\
$\forall_{i=0}^{n-1} (b[i] == 0) = (b[0]==0) \wedge (b[1]==0) \wedge ... \wedge (b[n-1]==0) $\\
\\
$\exists_{i=0}^{n-1} (b[i] == 0) = (b[0]==0) \vee (b[1]==0) \vee ... \vee (b[n-1]==0) $\\
\\
\\
\rule{\textwidth}{0.4mm}\\
\\
$\sum_{i=1}^{n}i^2 = ( \sum{i} : \mathbb{N}  \:\vert\:  1\leqslant i \leqslant n : i^2 ) $\\
\\
\tab \tab $( \sum{i} : \mathbb{N}, j:\mathbb{N}  \:\vert\:  1\leqslant i \leqslant 2 \wedge 1\leqslant j\leqslant 3 : i^j ) $\\
\\
\\
\rule{\textwidth}{0.4mm}\\
\\
$( \circ \:v_1 : T_1, ... , v_n : T_n \:\vert\: R : P ) $\\
\\
- $ \circ : T \times T \rightarrow T $ (wobei T ein Typ ist)\\
\\
Bsp: $+ : \mathbb{N} \times \mathbb{N} \rightarrow \mathbb{N}$\\
$+ (3,4) = 7$\\
\\
\\
ABELSCHES MONOID$\begin{cases} 
  a \circ b = b \circ a $ fuer alle $ a,b : T $ (Symmetrie)(Kommutativitaet)$\\
  MONOID\begin{cases}
    (a \circ b) \circ c = a \circ (b \circ c) $ fuer alle $ a,b,c : T $ (Assoziativitaet)$\\
    u \circ a = a = a \circ u\\
    $ es gibt ein $u : T$, so dass fuer alle $ a : T $ (neutrales Element)$\\
  \end{cases}
\end{cases}$\\
\\
\\
\rule{\textwidth}{0.4mm}\\
\\
\begin{tabular}{ l c r }
  $\circ$ & $T$ & $u$ \\
  $+\sum$ & $\mathbb{Z}$ & $0$ \\
  $*\prod$ & $\mathbb{Z}$ & $1$ \\
  $\forall$ & $\mathbb{B}$ & $true$\\
  $\exists$ & $\mathbb{B}$ & $false$
\end{tabular}
\\
\\
\rule{\textwidth}{0.4mm}\\
\\
String mit Konkatenation: Nicht-abelsches Monoid\\
$("a" + "b") + "c" \:equals\: "a" + ("b" + "c")$\\
$"a" + "" \:equals\: "a"$\\
$"a" + "b" \:!equals\: "b" + "a"$ (nicht equals)\\
\\
- $T_1, ... , T_n$ Datentypen\\
\\
- $V_1, ... , V_n$ Variablen\\
\\
alle paarweise verschieden\\
\\
$V_i$ vom Typ: $T_i$\\
\\
\\
- R : boolescher Ausdruck, kann $V_1 ... V_n$ enthalten, Bereich (Range)\\
\\
- P : beliebiger Ausdruck vom Typ T, kann $V_1 ... V_n$ enthalten, Koerper (Body)\\
\\
Typ der Quantifizierung : T\\
\\
\\
\rule{\textwidth}{0.4mm}\\
\\
$(\forall i : \mathbb{N} \:\vert\: 0\leqslant i \leqslant n : b[i]=0)$ und das Ganze ist : $\mathbb{B}$\\
$(\circ\: V_1 : T_1 \:\vert\: R : P )$ wobei $T_1 : \mathbb{N}, P : \mathbb{B}$\\  
\\
$\wedge : \mathbb{B} \times \mathbb{B} \rightarrow \mathbb{B}$\\
\\
$P : T_1 \times T_2 \times ... \times T_n \rightarrow T$\\
\\
\\
\rule{\textwidth}{0.4mm}\\
\\




\subsection{Semantik}

Bsp: $(+ i:\mathbb{Z} \:\vert\: -1\leqslant i\leqslant 2 : i^2)$\\
\\
\\
1. Fall (Topf $\neq \emptyset$)\\
\\
Von $\mathbb{Z}$ alle Zahlen ausfiltern (-1,0,1,2) (Menge)\\
\\
$\rightarrow^{1^2} ((-1)^2, 1^2, 0^2, 2^2) (1,1,0,4)$ (Multimenge)\\
\\
$\rightarrow 2^2 + 1^2 + (-1)^2 + 0^2$\\
\\
\\
2. Fall (Topf = 0)\\
\\
$\rightarrow$ Topf leer $\rightarrow$ Resultat: Neutrales Element (von +) $\rightarrow$ 0\\
\\
\\
\rule{\textwidth}{0.4mm}\\
\\
Beispiele:\\
\\
1) $(+ \:i:\mathbb{N} \:\vert\: 0\leqslant i < 4 : i*8) = (0*8) + (1*8) + ... $\\
\\
2) $(* \:i:\mathbb{N} \:\vert\: 0\leqslant i < 3 : i+1) = (0+1) * (1+1) * ... $\\
\\
3) $(\wedge \:i:\mathbb{N} \:\vert\: 0\leqslant i < 2 : i*d \neq 6) = ((0*d) \neq 6) \wedge ((1*d) \neq 6) \wedge ...  $\\
\\
4) $(\vee \:i:\mathbb{N} \:\vert\: 0\leqslant i < 21 : b[i]=0) = (b[0] == 0) \vee (b[1] == 0) \vee ...  $\\
\\
5) $(\sum k : \mathbb{N} \:\vert\: k^2 = 4 : k^2) = 2^2 = 4$\\
\\
6) $(\sum k : \mathbb{Z} \:\vert\: k^2 = 4 : k^2) = 2^2 + (-2)^2 = 8$\\
\\
\\
\\
\rule{\textwidth}{0.4mm}\\
\\


\section{Zweite Woche}

\subsection{Freie/Gebundene Variablen}

$( \circ \: v_1 : T_1,...,v_n : T_n \:\vert\: R : P )$\\
\\
\\
\rule{\textwidth}{0.4mm}\\
\\
E1: \tab$(\sum{i} : \mathbb{Z} \:\vert\: 0\leqslant i < n : i^2 )$\\
\\
- Wert haengt von $n$ ab, nicht von $i$\\
\tab $n = 3: \tab0\tab1\tab2$\\
\\
\tab\tab\tab $0^2 + 1^2 + 2^2 = 5$\\
\\
\tab $n = 0:$ kein $i$\\
\\
\tab\tab $0$ (neutral $+$)\\
\\
\\
\rule{\textwidth}{0.4mm}\\
\\

E2: \tab$(\sum{j} : \mathbb{Z} \:\vert\: 0\leqslant j < n : j^2 )$\\
\\
$n = 3 \rightarrow 5$\\
\\
$n = 0 \rightarrow 0$\\
\\
\\
\rule{\textwidth}{0.4mm}\\
\\
E3: $(\sum{i} \:\circled{1} : \mathbb{Z} \:\vert\: 0\leqslant i \:\circled{2} < n : i^2 \:\circled{3} ) + i \:\circled{4}$\\
\\
$(\leftarrow \rightarrow) :$ Gueltigkeitsbereich von $i$ (scope)\\
\\
$i$ tritt hier 4 mal auf (occurs)\\
\\
Auftreten (occurances) $\circled{1}, \circled{2}, \circled{3}$ gebunden\\
Auftreten $\circled{4}$ \uline{frei}\\
$\circled{2}$ und $\circled{3}$ gebunden an $\circled{1}$\\
$\circled{2}$ und $\circled{3}$ angewandte Auftreten (applied)\\
$\circled{1}$ bindende, deklarierende Auftreten (binding)\\
\\
\\
Eine Variable heisst \uline{frei} in einem Ausdruck E (expression), falls sie in E frei vorkommt.\\
\\
\\
$FV(E) =$ Menge der freie Variablen von E\\
\\
$FV(E_3) = \{'n', 'i'\}$ (Die Variablennamen und nicht die Werte der Variablen)\\
\\
\\
\rule{\textwidth}{0.4mm}\\
\\
$x, y : \mathbb{Z}$\\
$x = 3, y = 5$\\
$\{x, y\} = \{3, 5\}$\\
\\
$x = y = 3$\\
$\{x, y\} = \{3\}$\\
\\
$x + y * 2$\\
$y , 2 : *$ Operator\\
dann das Resultat mit $x$ und $+$ Operator\\
\\
\\
\rule{\textwidth}{0.4mm}\\
\\
E4: $(\sum{i} : \mathbb{Z} \:\vert\: 0\leqslant i< n : i^2 ) * (\sum{i} : \mathbb{Z} \:\vert\: 0\leqslant i< n : i^3 )$\\
\\
$FV(E_4) = \{'n'\}$\\
\\
\\
\rule{\textwidth}{0.4mm}\\
\\
E5: $( \prod{n} \:\vert\: k\leqslant n \leqslant l : (\sum{i} : \mathbb{Z} \:\vert\: 0\leqslant i< n : i^2 ) * (\sum{i} : \mathbb{Z} \:\vert\: 0\leqslant i< n : i^3 ) )$\\
\\
$FV(E_5) = \{'k', 'l'\}$\\
\\
\\
\rule{\textwidth}{0.4mm}\\
\\
E6: $(\sum{i} : \mathbb{Z} \:\vert\: 0\leqslant i\leqslant (\sum{i} : \mathbb{Z} \:\vert\: 2\leqslant i< 3 : i^2 ) : i^2 )$\\
\\
$FV(E_6) = \emptyset$\\
\\
Ein Ausdruck E ohne freie Variablen ($FV(E) = \emptyset$ oder $\{\}$) heisst geschlossen\\
\\
\\
\rule{\textwidth}{0.4mm}\\
\\
 $(\sum{i} : \mathbb{Z} \:\vert\: 1\leqslant i< 2 :  (\sum{j} : \mathbb{Z} \:\vert\: 1\leqslant j< 3 : i + j ) )$\\
\\
\uline{$i$ zuerst:}\\
\\
$i :$ \tab\tab\tab1 \tab\tab\tab\tab\tab\tab2\\
\tab $(\sum{j} : \mathbb{Z} \:\vert\: 1\leqslant j< 3 : 1 + j ) + (\sum{j} : \mathbb{Z} \:\vert\: 1\leqslant j< 3 : 2 + j )$\\
\\
$j :$ \tab\:1 \tab\:\:\:\:\:\:2 \tab\:\:\:\:\:\:3\tab\tab\:1 \tab\:\:\:\:\:\:\:2 \tab\:\:\:\:\:3\\
\tab$((1+1)+(1+2)+(1+3)) + ((2+1)+(2+2)+(2+3))$
\\
\\
\uline{$j$ zuerst:}\\
\\
$j :$ \tab\tab\tab\tab\tab\:\:\:\:\:1 \tab\:\:\:\:\:\:2 \:\:\:\:\:\tab3\\
\tab$(\sum{i} : \mathbb{Z} \:\vert\: 1\leqslant i< 2 :  ((i+1)+(i+2)+(i+3)) )$\\
\\
$i :$ \tab\tab\tab\:\:\:1 \tab\tab\tab\tab\tab\tab\:\:\:2\\
\tab$((1+1)+(1+2)+(1+3)) + ((2+1)+(2+2)+(2+3))$\\
\\
\\
\rule{\textwidth}{0.4mm}\\
\\
\\

\section{Dritte Woche}

\subsection{Saetze zur Quantifizierung}

\uline{Satz} (Dummy renaming)\\
\\
$(\circ \:v \:\vert\: R : P ) = (\circ \:w \:\vert\: R[v\leftarrow w] : P[v\leftarrow w] )$\\
\\
\uline{Voraussetzung:} $w \:\notin FV(R)\cup FV(P)$\\
\\
\uline{Dabei:} $E[v\leftarrow F]$ bezeichnet exakt denselben Ausdruck wie $E$, aber alle \uline{freien} Auftreten von $v$ ersetzt durch $(F)$.\\
\\
wobei $E, F:$ Ausdruck, $v:$ Variable\\
\\
\\
\rule{\textwidth}{0.4mm}\\
\\
Bsp: $(i+5)[i\leftarrow j+3] = (j+3)+5$\\
\\
wobei $(i+5): E$, $[i: v$, $j+3: F]$\\
\\
\\
$(i*5)[i\leftarrow j+3] = (j+3)*5$\\
\\
\\
$(\sum{i} \:\vert\: true : i^2)[i\leftarrow j+3] = (\sum{i} \:\vert\: true : i^2)$\\
\\
$(\sum{i} \:\vert\: true : i^2) = (\sum{j} \:\vert\: true : j^2)$\\
\\
$= (\sum{j} \:\vert\: true[i\leftarrow j] : i^2[i\leftarrow j])$\\
\\
$= (\sum{j} \:\vert\: true : j^2)$\\
\\
$42[i\leftarrow j+3] = 42$ "Man kann die Bedeutung des Universums nicht aendern."\\
\\
\\
\rule{\textwidth}{0.4mm}\\
\\
Es ist ein Unterschied, ob die Ersetzung innerhalb oder ausserhalb einer Quantifizierung angegeben wird.\\
\\
$(\sum{i} \:\vert\: true : i^2)[i \leftarrow j + 3]$\\
\\
Hier sollen alle freien Auftreten von Variable $i$ in $(\sum{i} \:\vert\: true : i^2)$ durch $j + 3$ ersetzt werden.\\
Aber alle Auftreten von $i$ sind in diesem Ausdruck gebunden, also ist nichts zu ersetzen.\\
\\
Dummy renaming sagt aus, dass wir die gebundenen Auftreten einer Variablen innerhalb einer Quantifizierung\\
konsistent umbenennen duerfen, solange wir dabei keine freien Variablen einfangen.\\
\\
$(\sum{i} \:\vert\: 1\leqslant i \leqslant n : i^2)$\\
$= (\sum{j} \:\vert\: 1\leqslant i \leqslant n[i \leftarrow j]: i^2[i \leftarrow j])$\\
$= (\sum{j} \:\vert\: 1\leqslant j \leqslant n : j^2)$\\
\\
Hier sind die Ersetzungen innerhalb der Quantifizierung.\\  
Und beachten Sie: im Teilausdruck $1\leqslant i\leqslant n$ ist die Variable $i$ frei,\\
daher liefert $1\leqslant i\leqslant n[i \leftarrow j]$ den Ausdruck $1\leqslant j\leqslant n$\\
\\
Im Gesamtausdruck $(\sum{i} \:\vert\: true : i^2)$ sind alle Auftreten von $i$ hingegen gebunden.\\
Aber in diesem Ausdruck wollen wir auch nicht ersetzen, sondern eben in den beiden Teilausdruecken.\\
\\
Ein Auftreten einer Variablen kann in einem Teilausdruck frei sein, aber im Gesamtausdruck gebunden.\\
Ob ein Auftreten frei oder gebunden ist, hängt immer vom betrachteten (Teil-)Ausdruck ab.\\
\\
\\
\rule{\textwidth}{0.4mm}\\
\\
Bsp: $(\sum{i} \:\vert\: 1\leqslant i\leqslant n : i^2 )$\\
\\
wobei $i: v$, $(1\leqslant i\leqslant n): R$, $i^2: P$\\
\\
$= (\sum{j} \:\vert\: (1\leqslant i\leqslant n)[i\leftarrow j] : i^2[i\leftarrow j] )$\\
\\
wobei $j: w$\\
\\
$= (\sum{j} \:\vert\: 1\leqslant j\leqslant n : j^2)$\\
\\
\\
\uline{Aber:} Vorsicht:\\
\\
$(\sum{i} : \:\vert\: 1\leqslant i\leqslant n : i^2 )$\\
$n=0 , \tab 0(neutral +)$\\
$n=1 , \tab 1$\\
\\
haengt von $n$ ab\\
\\
$\neq$\\
\\
$(\sum{n} : \:\vert\: 1\leqslant n\leqslant n : n^2 )$\\
$\infty$ undefiniert\\
\\
haengt \uline{nicht} von $n$ ab\\
\\
\\
\rule{\textwidth}{0.4mm}\\
\\

\section{Vierte Woche}

\subsection{Saetze zur Quantifizierung (Fortsetzung)}

$(\sum{i} \:\vert\: 0\leqslant i < n : i^2)[n\leftarrow n^2] = (\sum{i} \:\vert\: 0\leqslant i < n^2 : i^2)$\\
\\
$(\sum{i} \:\vert\: 0\leqslant i < n : i^2)[n\leftarrow i+1] \neq (\sum{i} \:\vert\: 0\leqslant i < i+1 : i^2)$ (geht nicht)\\
freies Auftreten von $i$ wird gefangen $\rightarrow$ name clash\\
\\
\\
$(\sum{i} \:\vert\: 0\leqslant i < n : i^2)[n\leftarrow i+1]$\\
$= (\sum{j} \:\vert\: 0\leqslant j < n : j^2)[n\leftarrow i+1]$\\
$= (\sum{j} \:\vert\: 0\leqslant j < i+1 : j^2)$\\
\\
\\
\rule{\textwidth}{0.4mm}\\
\\
\uline{Empty range}\\
\\
$(\circ \:v \:\vert\: false : P ) = u_\circ$ (Neutrales Element)\\
\\
\\
\rule{\textwidth}{0.4mm}\\
\\
\uline{One point}\\
\\
Voraussetzung: $v \:\notin FV(E)$\\
\\
$(\circ \:v \:\vert\: v = E : P ) = P[v\leftarrow E]$\\
\\
Bsp. $(\sum{i} \:\vert\: i = j + 3 : i^2) = i^2[i\leftarrow j+3] = (j+3)^2$\\
\\
\tab $(\sum{i} \:\vert\: i = j + i + 3 : i^2) \neq i^2[i\leftarrow j+i+3] = (j+i+3)^2$ (geht nicht)\\
\\
\\
\rule{\textwidth}{0.4mm}\\
\\
\uline{Split-off term}\\
\\
$(\circ \:i \:\vert\: 0\leqslant i < n+1 : P) = (\circ \:i \:\vert\: 0\leqslant i < n : P) \circ P[i\leftarrow n]$\\
\\
Bsp. $(\sum{i} \:\vert\: 0\leqslant i < n+1 : i^2) \tab = (\sum{i} \:\vert\: 0\leqslant i < n : i^2) + n^2$\\
\tab $0^2 + 1^2 + ... + (n-1)^2 + n^2 \:\:\:= (0^2 + 1^2 + ... + (n-1)^2) + n^2$\\
\\
$n=0 :$\\
\tab $(\circ \:i \:\vert\: 0\leqslant i < 1 : P) = (\circ \:i \:\vert\: 0\leqslant i < 0 : P) \circ P[i\leftarrow 0]$\\
\tab $i=0 :$\\
\tab \tab $P[i\leftarrow 0]$ (One point) $= u_\circ $(empty range)$  \:\circ P[i\leftarrow 0]$\\
\\
\\
\rule{\textwidth}{0.4mm}\\
\\

\subsection{Anwendung}

\uline{Praedikat}\\
\\
$i+1 > j : Bool$ macht Aussage ueber Werte von freien Variablen\\
\\
Feld $b[0...n-1]$ mit ganzen Zahlen$; n\geqslant 0$\\
\\
\\
\rule{\textwidth}{0.4mm}\\
\\
"$b$ enthaelt eine $-1$." $\rightarrow$ bedeutet mindestens\\
\\
$(\exists{i} : \mathbb{N} \:\vert\: 0\leqslant i < n : b[i] = -1)$\\
\\
\\
\rule{\textwidth}{0.4mm}\\
\\
"$b$ enthaelt genau eine $-1$."\\
\\
$(\exists{i} : \mathbb{N} \:\vert\: 0\leqslant i < n : (b[i] = -1) \wedge (\forall{j} : \mathbb{N} \:\vert\: (0\leqslant j < n) \wedge (j\neq i) : b[j] \neq -1 ))$\\
\\
$=$\\
\\
$1 = (\sum{i} : \mathbb{N} \:\vert\: (0\leqslant i < n) \wedge (b[i] = -1 : 1)$\\
\tab \tab \tab \tab \tab \:\:\:$\&\&$\\
\\
\\
\rule{\textwidth}{0.4mm}\\
\\
"$b$ enthaelt keine $-1$."\\
\\
$(\forall{i} : \mathbb{N} \:\vert\: 0\leqslant i < n : b[i] \neq -1)$\\
\\
$=$\\
\\
$\neg (\exists{i} : \mathbb{N} \:\vert\: 0\leqslant i < n : b[i] = -1) \rightarrow$ ($\neg$ ("$b$ enthaelt mindestens eine $-1$.")) \\
\\
\\
\rule{\textwidth}{0.4mm}\\
\\
$\neg (\exists{v} \:\vert\: R : P) = (\forall{v} \:\vert\: R : \neg P)$\\
$\neg (\forall{v} \:\vert\: R : P) = (\exists{v} \:\vert\: R : \neg P)$\\
\\
\\
\\
de Morgan\\
$\neg (\exists{v} \:\vert\: R : P) = \neg (P_0 \vee P_1 \vee ... \vee P_{n-1} \vee P_n) $\\
\tab \tab \tab $= ((\neg P_0) \wedge (\neg P_1) \wedge ... (\neg P_n))$\\
\tab \tab \tab $= (\forall{v} \:\vert\: R : \neg P)$\\
\\
\\
\rule{\textwidth}{0.4mm}\\
\\

\section{Fuenfte Woche}

\subsection{Magisches Quadrat}

\uline{Uebungsblatt 2, Aufgabe 3}\\
\\
$k,i: 1\leqslant k\leqslant n, 1\leqslant i\leqslant n$\\
\\
1) $(\exists{M} : \mathbb{N} \:\vert\: true : (\forall{i} \:\vert\: 1\leqslant i\leqslant n :(\sum{k} \:\vert\: 1\leqslant k\leqslant n : Q[i,k]) = M $\\
\tab\tab\tab\tab\tab\tab\tab $\wedge (\sum{k} \:\vert\: 1\leqslant k\leqslant n : Q[k,i]) = M$ \\
\tab\tab\tab\tab\tab\tab\tab $\wedge (\sum{k} \:\vert\: 1\leqslant k\leqslant n : Q[k,k]) = M$ \\
\tab\tab\tab\tab\tab\tab\tab $\wedge (\sum{k} \:\vert\: 1\leqslant k\leqslant n : Q[k,(n+1)-k)]) = M$\\
\tab\tab\tab\tab\tab\tab\tab $\wedge (\forall{m} : N \:\vert\: 1\leqslant m\leqslant n^2 :$\\
\tab\tab\tab\tab\tab\tab\tab\tab $(\exists{i,j} \:\vert\: 1\leqslant i < n\wedge 1\leqslant j < n: m = Q[i,j])))$ \\
\\
\\
2) $M = \frac{\sum_{i=1}^{n^2}i}{n}$\\
\\
$n * M = (\sum{i} \:\vert\: 1\leqslant i\leqslant n^2 : i )$\\
\\
$M = \frac{(\sum{i} \:\vert\: 1\leqslant i\leqslant n^2 : i )}{n} = \frac{n^2 * (n^2 + 1)}{2*n} = \frac{n * (n^2 + 1)}{2}$\\
\\
\\
\rule{\textwidth}{0.4mm}\\
\\

\subsection{Mathematische Induktion}

$(\mathbb{B}: Boolean)$\\
\\
Sei $P : \mathbb{N} \rightarrow \mathbb{B}$\\
zu zeigen:\\
$(\forall{n} : \mathbb{N} \:\vert\: true : P(n) )$\\
\\
\uline{Beispiel}\\
\\
$P(n): n^3 + 5*n$ ist ein Vielfaches von $6$\\
\\
$z$ ist Vielfaches von $6$ heisst:\\
$(\exists{i} : \mathbb{Z} \:\vert\: true : i*6  = z)$\\
\\
$0^3 + 5*0 = 0$(Zeuge) $*\:6$\\
$1^3 + 5*1 = 1*6$\\
$2^3 + 5*2 = 3*6$  (Muss bei allen $true$ zurueck geben!!)\\
\\
\\
\uline{Idee}: \uline{Induktionsprinzip}\\
\\
Man zeigt:\\
\\
1) $P(0)$\\
2) $P(n) \Rightarrow P(n+1)$ fuer alle $n : \mathbb{N}$\\
\\
\\
$P(0)$ gilt: Wegen 1)\\
\\
$(P(0) \wedge (P(0) \Rightarrow P(1)) \Rightarrow P(1)$\\
\tab wegen 2) mit $n=0$\\
\\
$(P(1) \wedge (P(1) \Rightarrow P(2)) \Rightarrow P(2)$\\
\tab wegen 2) mit $n=1$\\
\\
Damit gilt $P(n)$ fuer alle $n:\mathbb{N}$\\
\\
\\
\uline{Unser Beispiel}\\
\\
1) \uline{Induktionsanfang} (Base case)\\
zu zeigen: $P(0)$\\
$0^3 + 5*0 = 0$(Zeuge) $*\:6$\\
\\
2) \uline{Induktionsschritt} (inductive step)\\
zu zeigen: $P(n) \Rightarrow P(n+1)$ fuer alle $n:\mathbb{N}$\\
\\
Sei $n$ eine \uline{beliebige} natuerliche Zahl.\\
\\
\uline{Annahme}: Es gaelte $P(n) : n^3 + 5 * n$, dass heisst $n^3 + 5 * n = 6 * r$, mit $r : \mathbb{Z}$, ist vielfaches von $6$.\\
zu zeigen: (unter dieser Annahme) $P(n+1): (n+1)^3 + 5 * (n+1)$ ist vielfaches von $6$.\\
das heisst: $(n+1)^3 + 5 * (n+1) = 6 * s$, mit $s: \mathbb{Z}$\\
\\
$(n+1)^3 + 5 * (n+1)$\\
\\
\tab $<$Arith$>$ \\
\\
$= (n^3 + 3*n^2 + 3*n +1) + (5*n + 5)$\\
\\
\tab $<$Arith + Kaninchen$>$ \\
\\
$= (n^3 + 5*n) + (3*n^2 + 3*n + 6)$\\
\\
\tab $<$Annahme$>$\\
\\
$= 6*r + 3*n^2 + 3*n + 6$\\
\\
\tab $<$Arith + Kaninchen$>$\\
\\
$= 6*r + 3*n*(n+1) + 6$\\
\\
\tab $<$n*(n+1) ist gerade$>$\\
\\
$= 6*r + 3*(2*t) + 6$\\
\\
\tab $<$Arith$>$\\
\\
$= 6*(r+t+1)$ (Zeuge) \checkmark \\
\\
\\
\rule{\textwidth}{0.4mm}\\
\\
\uline{Modus ponens}\\
\\
\begin{tabular}{ l c r }
  $p$ & $q$ & $(p\wedge (p\Rightarrow q)) \Rightarrow q$ \\
  $f$ & $f$ & $w\:\:\:\:\:$ \\
  $f$ & $w$ & $w\:\:\:\:\:$ \\
  $w$ & $f$ & $w\:\:\:\:\:$ \\
  $w$ & $w$ & $w\:\:\:\:\:$
\end{tabular}
\\
\\
\\
\rule{\textwidth}{0.4mm}\\
\\
$(p\Rightarrow r)\wedge (\neg p \Rightarrow s)$\\
\\
$\equiv (p\wedge r) \vee (\neg p\wedge s)$\\
\\
\\
Sei $p$. Dann\\
\tab $(p\Rightarrow r)\wedge (\neg p \Rightarrow s)$\\
$=$\tab $r$ \:\:\:\:\:$\wedge$ \:\:\:$true$\\
$=$\tab $r$\\
\\
\tab $(p\wedge r)\vee (\neg p \wedge s)$\\
$=$\tab $r$ \:\:\:\:\:$\vee$ \:\:\:$false$\\
$=$\tab $r$\\
\\
Sei $\neg p$ Analog\\
\\
\\
\rule{\textwidth}{0.4mm}\\
\\

\section{Sechste Woche}

\subsection{Vollstaendige Induktion}

\uline{Arbeitsblatt 1 - Aufgabe 1}\\
\\
$(P(0)\wedge (\forall{n} : \mathbb{N} \:\vert : P(n) \Rightarrow P(n+1)) \Rightarrow (\forall{n} : \mathbb{N} \:\vert : P(n))$\\
\\
wobei $P(0)$ : Base Case\\
\tab $(\forall{n} : \mathbb{N} \:\vert : P(n) \Rightarrow P(n+1))$ : Iductive Case\\
\tab $(\forall{n} : \mathbb{N} \:\vert : P(n))$ : Ziel der Induktion\\
\\
\\
\uline{Complete Induction}\\
\\
$P(0)$\\
$(P(0) \wedge (P(0) \Rightarrow P(1))) \Rightarrow P(1)$\\
$(P(0) \wedge P(1) \wedge (P(0) \wedge P(1) \Rightarrow P(2))) \Rightarrow P(2)$\\
$(P(0) \wedge P(1) \wedge P(2) \wedge (P(0) \wedge P(1) \wedge P(2) \Rightarrow P(3))) \Rightarrow P(3)$\\
\\
$(P(0)\wedge (\forall{n} : \mathbb{N} \:\vert : (\forall{k} : \mathbb{N} \:\vert\: k\leqslant n : P(k)) \Rightarrow P(n+1)) \Rightarrow (\forall{n} : \mathbb{N} \:\vert : P(n))$\\
\\
\\
\uline{Aufgabe 2}\\
\\
Sei $k: \mathbb{N}$, $k\geqslant0$\\
\\
$(P(k)\wedge (\forall{n} : \mathbb{N} \:\vert\: n\geqslant k : P(n) \Rightarrow P(n+1)) \Rightarrow (\forall{n} : \mathbb{N} \:\vert\: n\geqslant k : P(n))$\\
\\
\\
\rule{\textwidth}{0.4mm}\\
\\
\uline{Fibonacci}\\
\\
$fib : \mathbb{N} \rightarrow \mathbb{N}$\\
\\
$\circled{A}$ $fib(0) = 0$\\
$\circled{B}$ $fib(1) = 1$\\
$\circled{C}$ $fib(n) = fib(n-1) + fib(n-2)$, $n\geqslant 2$\\
\\
\\
\uline{Satz} Fuer alle $n : \mathbb{N}$ gilt: \\
\\
$P(n): (\sum{i} : \mathbb{N} \:\vert\: 1\leqslant i\leqslant n : fib(i)) = f(n+2) -1$\\
\\
\uline{Beweis}\\
\\
1) Induktionsanfang: zu zeigen: $P(0)$, also\\
\\
$(\sum{i} : \mathbb{N} \:\vert\: 1\leqslant i\leqslant 0 : fib(i)) = f(0+2) -1$\\
\\
$(\sum{i} : \mathbb{N} \:\vert\: 1\leqslant i\leqslant 0 : fib(i))$\\
\\
\tab $<$ empty range, neutral + $>$\\
\\
$=0$\\
\\
\\
$fib(0+2) -1$\\
\\
\tab $<$ arith $>$\\
\\
$= fib(2) -1$\\
\\
\tab $<$ $\circled{C}$ mit $n=2$ $>$\\
\\
$= fib(0) + fib(1) -1$\\
\\
\tab $<$ $\circled{A}, \circled{B}$ $>$ \\
\\
$= 0 +1 -1$\\
\\
\tab $<$ arith $>$ \\
\\
$ =0$\\
\\
\\
2) Induktionsschritt: \\
\\
Sei $n$ eine \uline{beliebige} natuerliche Zahl\\
\\
\uline{Annahme:} $(\sum{i} : \mathbb{N} \:\vert\: 1\leqslant i\leqslant n : fib(i)) = fib(n+2) -1$\\
\\
\uline{zu zeigen:} $(\sum{i} : \mathbb{N} \:\vert\: 1\leqslant i\leqslant n+1: fib(i)) = fib(n+3) -1$\\
\\
\\
$(\sum{i} : \mathbb{N} \:\vert\: 1\leqslant i\leqslant n +1  : fib(i))$\\
\\
\tab $<$ range split (split-off term) $>$ \\
\\
$= (\sum{i} : \mathbb{N} \:\vert\: 1\leqslant i\leqslant n : fib(i)) + fib(n+1)$\\
\\
\tab $<$ Annahme $>$ \\
\\
$= fib(n+2) -1 + fib(n+1)$\\
\\
\tab $<$ arith $>$\\
\\
$= (fib(n+2) + fib(n+1)) -1$\\
\\
\tab $<$ $\circled{C}$, mit $n+3 \geqslant 2$ $>$ \\
\\
$= fib(n+3) -1 $\\
\\
$\qed$\\
\\
\\
\rule{\textwidth}{0.4mm}\\
\\
\uline{Satz:}\\
\\
Fuer alle $n:\mathbb{N}$ mit $n\geqslant3$ gilt:\\
\\
$2n+1 < 2^n$\\
\\
\uline{Beweis}\\
\\
1) \uline{IA}\\
\\
$2*3 + 1 < 2^3 \equiv 7<8$ \checkmark \\
\\
2) \uline{IS}\\
\\
Sei $n$ eine \uline{beliebige} natuerliche Zahl mit $n\geqslant 3$\\
\uline{Annahme} Es gelte: $2n+1 < 2^n$\\
\uline{zu zeigen:} Es gilt: $2(n+1)+1 < 2^{(n+1)}$\\
\\
\uline{Beweis}\\
\\
$2(n+1) + 1$\\
\\
\tab $<$ arith $>$ \\
\\
$= (2n +1) +2 $\\
\\
\tab $<$ Annahme $>$ \\
\\
$\Rightarrow (2n+1) + 2 < 2^n +2$\\
\\
$\leqslant$\\
\\
$2^n + 2^n $\\
\\
\tab $<$ arith $>$ \\
\\
$= 2^n +1$ \checkmark \\
\\
\\
\rule{\textwidth}{0.4mm}\\
\\
$2*2* ... *2 = 2^n$\\
($n$ mal)\\
\\
$pow1(n) = (\prod{i} : \mathbb{N} \:\vert\: 1\leqslant i\leqslant n : 2)$\\
\\
$pow2(n)\begin{cases}
    2^0 = 1 \\
    2^n = 2 * 2^{n-1}$, fuer $n>0\\
  \end{cases}$\\
\lstinputlisting[language=Java]{pow1pow2.java}

\uline{Satz:} Fuer alle $n: \mathbb{N}$ gilt: $pow1(n) = pow2(n)$\\
\\
\uline{Beweis} Aufgabe!\\
\\
\\
\rule{\textwidth}{0.4mm}\\
\\
$x<y \Rightarrow x\leqslant y$\\
\\
$x\leqslant y \Rightarrow x<y \vee x=y$\\
\\
$p \Rightarrow p\vee q$\\
\\
$x\leqslant y \nRightarrow x<y$\\
\\
$15\leqslant 15 \nRightarrow 15<15$\\
\\
\rule{\textwidth}{0.4mm}\\
\\

\section{Siebte Woche}

\subsection{Zahlentheorie}

\uline{Teilbarkeit}\\
\\
$\mathbb{Z} = \{...,-2,-1,0,1,2,...\}$, $c,b : \mathbb{Z}$\\
\\
$c\backslash b \equiv (\exists{k} : \mathbb{Z} \:\vert : b = k * c)$\\
\\
$c\backslash b$ : "c teilt b"\\
\tab\: "b ist teilbar durch c"\\
\tab\: "c ist Teiler von b"\\
\tab\: "b ist Vielfaches von c"\\
\\
\\
\uline{Beispiel:}\\
\\
$7\backslash 13 = false$\\
$(-7)\backslash 14 = true$\\
nicht einheitlich in Literatur $\begin{cases}
    0\backslash 14 = false $ (es existiert kein $k) \\
    0\backslash 0 = true  \\
  \end{cases}$\\
\\
\\
\uline{Satz} $b,c,d : \mathbb{Z}$\\
\\
(1) $c\backslash c$ (Reflektivitaet)\\
(2) $c\backslash 0$\\
(3) $1\backslash b$\\
(4) $c\backslash 1 \Rightarrow c=1 \vee c=-1$\\
(5) $d\backslash c \wedge c\backslash b \Rightarrow d\backslash b$ (Transitivitaet)\\
(6) $b\backslash c \wedge c\backslash b \Rightarrow b=c \vee b =-c$\\
\tab auf $\mathbb{Z}$ also nicht antisymmetrisiert, auf $\mathbb{N}$ aber schon\\
(7) $b\backslash c \Rightarrow b\backslash (c*d)$\\
(8) $b\backslash c \Rightarrow (b*d)\backslash (c*d)$\\
(9) $1<b \wedge b\backslash c \Rightarrow \neg (b\backslash(c+1))$\\
\\
(1) und (5) und (6) heisst: $\backslash$-Relation ist eine partielle Ordnung auf $\mathbb{N}$ (aber nicht auf $\mathbb{Z}$)\\
\\
\uline{Beweis:}\\
\\
(1) zu zeigen: es gibt $k : \mathbb{Z}$ mit $c = k*c$\\
\tab Waehle $k=1 : \mathbb{Z}$\\
(2) $0 = k*c$ , $ k=0 : \mathbb{Z}$\\
(3) $b = k*1$ , $ k=b : \mathbb{Z}$\\
(4) $1 = k*c = 1*1 = (-1)*(-1) \Rightarrow c=1 \vee c=-1$\\
(5)\\ 
$d\backslash c$, also gibt es $k_1 : \mathbb{Z}$ mit $c = k_1 * d$\\
$c\backslash b$, also gibt es $k_2 : \mathbb{Z}$ mit $b = k_2 * c$\\
aber $b=k_2 *c = k_2*(k_1*d) = (k_2 * k_1) *d$\\
also gibt es $k=k_1*k_2: \mathbb{Z}$ mit $b=k*d$, also $d\backslash b$\\
(6)\\
$b\backslash c$, also gibt es $k_1 : \mathbb{Z}$ mit $c = k1 * b$\\
$c\backslash b$, also gibt es $k_2 : \mathbb{Z}$ mit $b = k2 * c$\\
also $b = k_2 * c = k_2*(k_1*b) = (k_2*k_1)*b$\\
also $b- (k_2*k_1)*b =0$\\
also $b* (1- k_2*k_1) =0$\\
also $b=0 \vee 1- k_2*k_1 =0$\\
also $b=0 \vee k_2 * k_1 = 1$\\
also $b=0 \vee k_2 = k_1 = 1 \vee k_2 = k_1 = -1$\\
also $b=0 \vee c = b \vee c =-b $\\
mit $b=0$ ist $c =b$\\
\\
\\
\rule{\textwidth}{0.4mm}\\
\\
\uline{Satz} Seien $a,b,c : \mathbb{Z}$ \\
\\
Dann \: $a\backslash b \wedge a\backslash c \Rightarrow a\backslash (b+c)$\\
\\
$a\backslash b \Rightarrow b = k_1 * a$\\
\\
$a\backslash c \Rightarrow c = k_2 * a$\\
\\
$a\backslash (b+c) \Rightarrow a\backslash (k_1 * a +  k_2 * a) \Rightarrow a\backslash a*(k_1+k_2)$ \checkmark    (Satz (7)) \\
\\
\\
\uline{Beweis}\\
\\
\uline{Annahme 1:} $a\backslash b$, also nach Definition existiert $k_1 : \mathbb{Z}$ mit $b = k_1 *a$\\
\\
\uline{Annahme 2:} $a\backslash c$, also nach Definition existiert $k_2 : \mathbb{Z}$ mit $c = k_2 *a$\\
\\
\uline{zu zeigen:} Es existiert $k_3 : \mathbb{Z}$ mit $b+c = k_3 * a$\\
\\
\\
$b+c$\\
\\
\tab $<$ Annahme 1 und 2 $>$\\
\\
$= k_1 * a + k_2 * a$\\
\\
\tab $<$ arith $>$\\
\\
$= (k_1 + k_2) *a$\\
\\
\tab $<$ mit $k_3 = k_1 + k_2 : \mathbb{Z}$ $>$\\
\\
$= k_3 * a$\\
\\
$\qed$\\
\\
\\
\rule{\textwidth}{0.4mm}\\
\\

\section{Achte Woche}

\subsection{Euklidische Division}

\uline{Satz} (Euklidische Division auf $\mathbb{N}$)\\
\\
Seien $a,b : \mathbb{N} $, $ b\neq 0$.\\
\\
Dann gibt es eindeutige $q, r : \mathbb{N}$ mit\\
\\
$a = b*q + r \wedge 0\leqslant r < b$\\
\\
$\qed$\\
\\
\\
\uline{Satz} (Euklidische Division auf $\mathbb{Z}$)\\
\\
Seien $a,b : \mathbb{Z} $, $ b\neq 0$.\\
\\
Dann gibt es eindeutige $q : \mathbb{Z}$, $ r : \mathbb{N}$ mit\\
\\
$a = b*q + r \wedge 0\leqslant r < |b|$\\
\\
$\qed$\\
\\
\\
\rule{\textwidth}{0.4mm}\\
\\
$a,b : \mathbb{N}$, $b\neq 0$\\
\\
$a = 17$, $b = 5$\\
\\
$17 = 5*0 + 17 \wedge 0\leqslant 17$\\
\\
$17 = 5*1 + 12 \wedge 0\leqslant 12$\\
\\
$17 = 5*2 + 7 \wedge 0\leqslant 7$\\
\\
$17 = 5*3 + 2 \wedge 0\leqslant 2$\\
\\
$a = b * q + r \wedge 0\leqslant r$ (Invariante)\\
\\
Stop bei $r<b$\\
\\
$q, r$ seien Variablen einer imperativer Programmiersprache (keine Mathematische Variablen)
\\
\\
\rule{\textwidth}{0.4mm}\\
\\
$\{ a\geqslant 0 \wedge b> 0 \}$\\
\\
VC3 $\Rightarrow$\\
\\
$\{ a = b * 0 + a \wedge 0\leqslant a \} $\\
\\
$q, r := 0, a;$\\
\\
$\{ a = b * q + r \wedge 0\leqslant r \} $\\
\\
\uline{while} $r\geqslant b$ \uline{do}\\
\\
$\{ a = b * q + r \wedge 0\leqslant r \wedge r\geqslant b \} $\\
\\
VC1 $\Rightarrow$\\
\\
$\{ a = b * (q+1) + (r-b) \wedge 0\leqslant r-b \} $ (Precondition)\\
\\
\tab $q,r := q+1, r-b$\\
\\
$\{ a = b * q + r \wedge 0\leqslant r \} $\\
\\
\uline{endwhile}\\
\\
$\{ a = b * q + r \wedge 0\leqslant r \wedge \neg r \geqslant b \} $\\
\\
VC2 $\Rightarrow$\\
\\
$\{ a = b * q + r \wedge 0\leqslant r < b\} $\\
\\
\\
VC: Verification Condition\\
\\
\\
VC3: $a = b* 0 + a$  $(true)$\\
\\
\tab $a\geqslant 0 \wedge b > 0 \Rightarrow true \wedge 0 \leqslant a$\\
\\
\\
VC2: $\neg r \geqslant b \equiv  r< b$\\
\\
\\
VC1: $a= b(q+1) + (r-b) $\\
$\Leftarrow$\\
$a = b*q + b + r -b$\\
$\Leftarrow$\\
$a = b*q +r$\\
\\
(von unten nach oben)\\
\\
\\
Warum gilt $0\leqslant r-b $?\\
\\
Weil $r\geqslant b$!\\
\\
$r\geqslant b \Rightarrow$ $ r-b \geqslant 0 \Rightarrow$ $ 0\leqslant r-b$\\
\\
\\
\rule{\textwidth}{0.4mm}\\
\\
$\{ x+1 > 5 \}$ (Vorbedingung)\\
$x := x+1 $\\
$\{ x > 5 \}$ (Nachbedingung)\\
\\
\\
\rule{\textwidth}{0.4mm}\\
\\

\subsection{Fundamentaler Schleifen-Satz}

\uline{Satz} (Fundamentaler Schleifen-Satz)\\
\\
Seien $B$ und $I$ boolesche Ausdruecke und $C$ ein Kommando.\\
\\
Es gelte $\{ I \wedge B \}$ $C$ $\{ I \}$, d.h. $I$ ist eine Invariante der Schleife \uline{while} $B$ \uline{do} $C$ \uline{end}.\\
\\
Die Ausfuehrung der Schleife beginne im Zustand der $I$ erfuellt. Dann gilt $I$ nach jedem Schleifendurchlauf.\\
\\
$\qed$\\
\\
\\
\uline{Beweis}\\
\\
Durch Induktion. Wir zeigen, dass "Invariante ist nach $n$ Schleifendurchlaeufe erfuellt" fuer alle $n \in M$.\\
Dabei ist $M = N$, falls $\infty$-Schleife, und $M = \{ 0 ... N \} $ mit $N : \mathbb{N}$ Anzahl der Schleifendurchlaeufe (SDL).\\
\\
\uline{IA} $I$ ist nach $0$ SDL erfuellt, nach Voraussetzung\\
\\
Sei $M=N$ (Fall 2: Sei $M = \{ 0 ... N \} $)\\
\\
\uline{IS} Sei $n$ eine \uline{beliebige} natuerliche Zahl. (Fall 2: $n \leqslant N-1$)\\
\\
Annahme: Es gelte: $I$ ist nach $n$ SDL erfuellt.\\
zu zeigen: Es gilt: $I$ ist nach $n+1$ SDL erfuellt.\\
\\
\uline{Beweis}\\
\\
$I$ ist nach $n$ SDL erfuellt. Weiterer SDL bedeutet, dass Schleifenbedingung $B$ gilt.\\
$I \wedge B$ gilt also, also nach Ausfuehrung von $C$ wieder $I$, wegen $\{ I \wedge B \}$ $C$ $\{ I \} $, also\\
$I$ nach $n+1$ SDL erfuellt.\\
\\
$\qed$\\
\\
\\
\uline{Satz} Falls Ausfuehrung terminiert, gilt am Ende $I \wedge \neg B$\\
\\
\uline{Beweis}\\
\\
Nach obigem Satz, gilt $I$ am Ende. Die Schleife terminiert, gilt auch $\neg B$.\\
\\
\\
\rule{\textwidth}{0.4mm}\\
\\
$\{ I \wedge B \}$ $C$ $\{ I \}$\\
\\
$\{I \}$ \uline{while} $\{ B \} $ \uline{do} $C$ \uline{end} $\{ I \wedge \neg B \} $\\
\\
\\
\rule{\textwidth}{0.4mm}\\
\\

\section{Neunte Woche}

\subsection{Euklid auf $\mathbb{Z}$, $\mathbb{N}$}

\uline{Satz} (Euklid auf $\mathbb{Z}$)\\
\\
$a, b : \mathbb{Z}$, $b \neq 0$. Dann gibt es eindeutige $q: \mathbb{Z}$, $r : \mathbb{N}$ mit\\
\\
$a = b* q + r \wedge 0 \leqslant r < |b|$\\
\\
$\qed$\\
\\
\\
\rule{\textwidth}{0.4mm}\\
\\
\uline{Truncated Division}\\
\\
$a = b* q + r \wedge \begin{cases}
    0 \leqslant r < |b|,\:\:\:   $ if $a\geqslant 0  \\
     -|b| < r \leqslant 0 $, if $a\leqslant 0 \\
  \end{cases}$\\
\\
$r: \mathbb{Z}$\\
\\
\\
\rule{\textwidth}{0.4mm}\\
\\
\uline{Satz} (Euklid auf $\mathbb{N}$)\\
\\
$a, b : \mathbb{N}$, $b \neq 0$. Dann gibt es eindeutige $q: \mathbb{N}$, $r : \mathbb{N}$ mit\\
\\
$a = b* q + r \wedge 0 \leqslant r < b$\\
\\
$\qed$\\
\\
\\
\uline{Beweis} fuer Euklid auf $\mathbb{Z}$\\
\\
\uline{4 Faelle}\\
\\
\\
\uline{1. Fall} $a\geqslant 0$, $b > 0$\\
\\
Trivial\\
\\
\\
\uline{2. Fall} $a\geqslant 0$, $b < 0$\\
\\
Euklid $\mathbb{N}$ anwenden auf $(a, -b)$. Liefert\\
\\
$a = (-b)*q + r \wedge 0 \leqslant r < -b$\\
\\
Waehle $(q', r') = (-q, r)$ \tab $-b = |b|$\\
\\
$a = b*(-q) + r \wedge 0 \leqslant r < |b|$\\
\\
$a = b*q' + r' \wedge 0 \leqslant r' < |b|$\\
\\
\\
\uline{3. Fall} $a< 0 \wedge b > 0 $\\
\\
Euklid $\mathbb{N}$ anwenden auf $(-a,b)$ Liefert $q, r$ mit\\
\\
$-a = b*q + r \wedge 0 \leqslant r < b$\\
\\
\\
\uline{Sei $r=0$} Waehle $(q', r') = (-q, 0)$ \tab  $b = |b|$\\
\\
$a = b*(-q) - r \wedge 0 \leqslant r < |b|$\\
\\
$a = b*q' + r \wedge 0 \leqslant r < |b|$\\
\\
\\
\uline{Sei $1\leqslant r \leqslant b-1$} Waehle $(q', r') = (-(q+1), b-r)$\\
\\
$-a = b*q + r$\\
\\
$a = b*(-q) -r$\\
\\
$= b*(-q) -b+b -r$\\
\\
$= b*(-q-1) +(b-r)$\\
\\
$= b*(-(q+1)) +(b-r)$\\
\\
$= b*q' + r'$\\
\\
\\
$1\leqslant \underbrace{b-r}_\text{$r'$} \leqslant b-1 \begin{cases} 
                               1\leqslant r \rightarrow -r\leqslant -1 \rightarrow b-r \leqslant b-1\\
				r \leqslant b-1 \rightarrow 1 \leqslant b-r\\
                              \end{cases}$\\
\\
\\         
\uline{4. Fall} $a< 0$, $b<0$ (??)\\
\\
Aehnlich\\
\\
\\
\rule{\textwidth}{0.4mm}\\
\\

\section{Zehnte Woche}

\subsection{Eindeutigkeit}

\uline{Satz}\\
\\
Seien $a,b: \mathbb{Z}$, $b \neq 0$. Dann gibt es eindeutige $q,r : \mathbb{Z}$ mit\\
\\
$a = b * q + r \wedge 0 \leqslant r < |b|$\\
\\
\\
\uline{Eindeutigkeit} Seien $a,b,q,r,q',r': \mathbb{Z}$ mit $b \neq 0$\\
\\
Es gelte:\\
\tab $a = b * q + r \wedge 0 \leqslant r < |b|$\\
\tab $a = b * q' + r' \wedge 0 \leqslant r' < |b|$\\
\\
Dann gilt $q=q'$ unr $r=r'$.\\
\\
\uline{Beweis:}\\
\\
$a = b * q + r$\\
$a = b * q' + r'$\tab $-$\\
\rule{2cm}{0.2mm}\\
$0=b(q-q') + (r-r')$\\
$\rightarrow |b| |q-q'| = |r-r'|$\\
\\
$0 \leqslant r' < |b|$\\
$0 \geqslant -r' < -|b|$\\
\rule{2cm}{0.2mm}\\
\\
$-|b| < -r' \leqslant 0$\\
$0 \leqslant r < |b|$\tab $+$\\
\rule{2cm}{0.2mm}\\
$-|b| < r-r' < |b|$ \tab $\rightarrow$ echt kleiner $\circled{$<$}$ !\\
\\
$\rightarrow |r-r'| < |b|$\\
$|b| |q-q'| < |b|$\\
\\
$\rightarrow |q-q'| < 1$\\
\\
$\rightarrow q-q' = 0$\\
\\
$\rightarrow q = q'$\\
\\
also auch $r=r'$\\
\\
\\
\uline{Definition} Seien $a,b,q,r : \mathbb{Z}$ mit $b \neq 0$\\
Es gelte:\\
\tab $a = b * q + r \wedge 0 \leqslant r < |b|$\\
\\
Nach Satz $q,r$ \uline{eindeutig}\\
\\
Definiere Funktion:\\
\\
$div_E, mod_E : \mathbb{Z} \times \mathbb{Z}^{\neq 0} \rightarrow \mathbb{Z}$\\
\\
$a\: div_E\: b = div(a,b) = q$\\
\\
$a\: mod_E\: b = mod(a,b) = r$\tab ($E \rightarrow$ Euklid)\\ 
\\
$div = div_E$\\
\tab hier im Kurs!\\
$mod = mod_E$\\
\\
\\
\rule{\textwidth}{0.4mm}\\
\\

\subsection{GCD (Greatest Common Divisor)}

\uline{Def} Seien $a,b : \mathbb{Z}$\\
\\
$D_{a,b} = \{ d : \mathbb{Z} \:\vert\: d\backslash a \wedge d\backslash b\}$\\
\\
\uline{Beispiele}
\\
$D_{5,14}= \{-5,-1,1,5\} \wedge \{-14,-7,-2,-1,1,2,7,14\}$\\
\\
$= \{-1,1\}$\\
\\
\\
$D_{3,0}= \{-3,-1,1,3\} \wedge \mathbb{Z} = \{-3,-1,1,3\}$\\
\\
$= D_{-3,0}$\\
\\
\\
$D_{0,0} = \mathbb{Z} \cap \mathbb{Z} = \mathbb{Z}$ \tab ($0\backslash 0$ bei uns!)\\
\\
\\
\uline{Satz} Seien $a,b : \mathbb{Z}$\\
\\
(1) $1 \in D_{a,b}$, also $D_{a,b} \neq \emptyset$\\
\\
(2) $a\neq 0 \wedge d \in D_{a,b} \Rightarrow |d| \leqslant |a|$\\
\tab $b\neq 0 \wedge d \in D_{a,b} \Rightarrow |d| \leqslant |b|$\\
\tab $a\neq 0 \wedge b\neq 0 \wedge d \in D_{a,b} \Rightarrow |d| \leqslant min(|a|,|b|)$\\
\\
\\
\uline{Korollar} (Folgesatz)\\
\\
Seien $a,b : \mathbb{Z}$ mit $a\neq 0 \vee b\neq 0$\\
\\
Dann hat $D_{a,b}$ groesstes Element (wegen (1) hat es ueberhaupt ein Element, wegen (2) ist jedes Element durch $|a|$ bzw. $|b|$ begrenzt)\\
\\
$\qed$\\
\\
\\
Def (GCD) Seien $a,b : \mathbb{Z}$ mit $a\neq 0 \vee b\neq 0$\\
\\
$gcd(a,b) = a\: gcd\: b = max(D_{a,b}) = $\\
\\
$(max\: d: \mathbb{Z} \:\vert\: d\backslash a \wedge d\backslash b\ : d)$\\
\\
$gcd(0,0) = 0 \leftarrow$ (selber so definiert - nicht einheitlich)\\
\\
\\
\rule{\textwidth}{0.4mm}\\
\\
\\
\\
$a\leqslant b$\\
$c\leqslant d$\tab $+$\\
\rule{2cm}{0.2mm}\\
$a+c\leqslant b+d$\\
\\
\\
$a\leqslant b$\\
$c < d$\tab $+$\\
\rule{2cm}{0.2mm}\\
$a+c < b+d$\tab ($\circled{$<$}$ !)\\
\\
$<$ ist in diesem Fall wertvoller fuer den Beweis und deshalb behalten wir das so.
\\
\\
\rule{\textwidth}{0.4mm}\\
\\
\uline{GCD} $a,b:\mathbb{Z}$\\
\\
\uline{Satz}\\
\\
$gcd(a,0) = |a|$\\
$gcd(0,b) = |b|$\tab $0$ ist neutrales Element von $gcd$\\
$gcd(a,a) = |a|$\\
$a\: gcd \: b = b\: gcd \: a$\tab Symmetrie\\
$gcd(a,1) = 1$\tab $1$ ist Null von $gcd$ (Destruktor)\tab \tab ($a * 0 = 0$) ($a\: gcd \: 1 = 1$)\\
$gcd(a,b) = gcd(a\: mod\: b,b)$\\
$a\backslash b \Rightarrow gcd(a,b) = a$\\
$gcd(0,0) = 0$\\
$gcd(a,b) = gcd(|a|,|b|)$, $k:\mathbb{Z}$\\
\\
\\
$gcd(a,b) = gcd(a+k*b,b)$ \tab $k:\mathbb{Z}$\\
$gcd(a,b) = gcd(a,b + k*a)$\\
$gcd(a,b) = gcd(a-k*b,b)$\\
\\
\\
\uline{Beweis}\\
\\
1. Fall: $a=b=0$ \tab $a-k*b =0$  Trivial!\\
\\
2. Fall: $a\neq 0 \vee b\neq 0$\\
\\
Wir zeigen: $D_{a,b} = D_{a-k*b,b}$ damit auch GCDs gleich\\
\\
$d\in D_{a,b}$\\
\\
\tab $<$ Def. $D_{a,b}$ $>$\\
\\
$\Rightarrow d\backslash a \wedge d\backslash b$\\
\\
\tab $<$ Def. $\backslash$ mit $k_1,k_2 : \mathbb{Z}$ $>$\\
\\
$\Rightarrow a = k_1*d \wedge b = k_2*d$\\
\\
\tab $<$ Einsetzen fuer $a-k*b$ $>$\\
\\
$\Rightarrow a-k*b = k_1 *d - k*k_2*d \wedge b = k_2 *d$\\
\\
\tab $<$ Arith. $>$\\
\\
$\Rightarrow a -k*b = d(k_1 - k_2*k) \wedge b = k_2 *d$\\
\\
\tab $<$ Def. $\backslash$ mit $k_1 - k*k_2: \mathbb{Z}$ $>$\\
\\
$\Rightarrow d \backslash (a-k*b) \wedge d\backslash b$\\
\\
\tab $<$ Def. $D_{a-k*b,b}$ $>$\\
\\
$\Rightarrow d \in D_{a-k*b,b}$\\
\\
\\
$1/2 \qed$\\
\\
\\
Bisher gezeigt: $D_{a,b} \subseteq D_{a-k*b,b}$\\
\\
Aufgabe: Zeigen Sie $D_{a-k*b,b} \subseteq D_{a,b}$
\\
\\
$\qed$\\
\\
\\
\uline{Satz} $b\backslash a \Rightarrow r=0$\\
\\
\uline{Beweis} $b\backslash a$, also $\exists{k} : \mathbb{Z}$ mit $a=k*b$\\
\\
ausserdem $a=b*q+r \wedge 0 \leqslant r < |b|$ (*)\\
mit \uline{eindeutige} $q,r : \mathbb{Z}$\\
\\
Setze $q=k$ und $r=0$. Diese erfuellt Bedingung (*)\\
\\
Da $q,r$ eindeutig, folgt $r=0$.\\
\\
\uline{Beweis 2}\\
\\
$b \backslash a$, also $\exists{k}:\mathbb{Z}$ mit $ a = k*b$\\
\\
ausserdem ist $a = b*q+r \wedge 0 \leqslant r < |b|$\\
also\\
\tab $k*b = b*q+r$\\
also\\
\tab $r = b(k-q) < |b|$\\
\\
$r = b(k-q) \geqslant 0$\\
also\\
\tab $b(k-q) = |b(k-q)| = |b| |k-q|$\\
\\
$r = b(k-q) = |b| |k-q| < |b|$\\
\\
also mit $b \neq 0$ und $|b| > 0$ gilt\\
\\
\tab $|k-q| < 1$\\
also\\
\tab $k-q = 0$\\
also\\
\tab $k=q$\\
also\\
\tab $r=0$\\
\\
$\qed$\\
\\
\\
$a\backslash b$, also $\exists{k_1} : \mathbb{Z}$ mit $b = k_1 * a$\\
\\
\\
\rule{\textwidth}{0.4mm}\\
\\

\section{Elfte Woche}

\subsection{Euklidischer Algorithmus}

$\{a>0 \wedge b>0\}$\\
\\
$(x,y) := (a,b);$\\
\\
\uline{while} $x\neq y$ \uline{do}\\
\\
\tab \uline{invariante} $gcd(x,y) = gcd(a,b) \wedge x>0 \wedge y>0$\\
\\
\uline{if} $x>y$ \uline{then}\\
\\
\tab $x:= x-y$\\
\\
\uline{else} \tab// $x<y$ (assert $x<y$)\\
\\
\tab $y := y-x$\\
\\
\uline{endif}\\
\\
\uline{endwhile}\\
\\
postcondition: $\{ x = y = gcd(a,b)\}$\\
\\
\\
\uline{Beweis}\\
\\
$\{ (gcd(x,y) = gcd(a,b))^{\circled{A}} \wedge (x>0)^{\circled{B}} \wedge (y>0)^{\circled{C}} \wedge (x\neq y)^{\circled{D}} \wedge (x> y)^{\circled{E}} \}$\\
\\
$\Rightarrow$\\
\\
$\{ (gcd(x-y,y) = gcd(a,b))^{\circled{1}} \wedge (x-y>0)^{\circled{2}} \wedge (y>0)^{\circled{3}} \}$\\
\\
$x := x-y$\\
\\
$\{ (gcd(x,y) = gcd(a,b)) \wedge (x>0) \wedge (y>0)\}$\\
\\
\\
\\
\\
$\circled{1}\: gcd(x,y) = gcd(a,b)\: \circled{A}$\\
\\
\tab $<$ $gcd(a,b) = gcd(a-k*b,b)$ $>$\\
\\
$\Rightarrow gcd(x-1*y,y) = gcd(a,b)$\\
\\
\\
$\circled{2} \:x>y\: \circled{E}$\\
\\
$\Rightarrow x-y>0$\\
\\
\\
$\circled{3} = \circled{C}$\\
\\
\\
\rule{\textwidth}{0.4mm}\\
\\
\uline{Inv} $\{a>0 \wedge b>0\}$\\
\\
$\{ gcd(a,b) = gcd(a,b) \wedge a>0 \wedge b>0$\\
\\
$(x,y) := (a,b)$\\
\\
$\{ gcd(x,y) = gcd(a,b) \wedge x>0 \wedge y>0$\\
\\
\\
\rule{\textwidth}{0.4mm}\\
\\
$\{ gcd(x,y) = gcd(a,b) \wedge x>0 \wedge y>0 \wedge x=y$\\
\\
$\Rightarrow x=y=gcd(a,b)$\\
\\
$gcd(x,y) = gcd(x,x) = x = gcd(a,b)\: \checkmark$\\
\\
\\
\rule{\textwidth}{0.4mm}\\
\\

\section{Zwoelfte Woche}

\subsection{GCD langsam-schnell}

$x \: mod \: y =0 \rightarrow$ Stop bei schnellem Algorithmus. $gcd(a,b) = gcd(0,y) = y$,\tab $x,y\geqslant0$\\
\\
$x=y \rightarrow$ Stop bei langsamen Algorithmus. $gcd(a,b) = gcd(x,x) = x$, \tab$x,y>0$\\
\\
$gcd(a,b) =^{\circled{*}} gcd(b, a\: mod\: b)$\\
\\
\\
$a,b,k : \mathbb{Z}$\\
\\
$gcd(a,b) =^{\circled{1}} gcd(a-k*b, b)$\\
\\
\\
Sei $b\neq0$\\
\\
$a = b* q + r \wedge 0 \leqslant r < |b|$,\tab $q = a\: div \:b$,\tab $r = a\: mod\: b$\\
\\
$a = b*(a\: div \:b) + (a\: mod\: b)$\\
\\
$a\: mod\: b =^{\circled{2}} a - b*(a\: div\: b)$\\
\\
\\
$gcd(a,b)$\\
\\
\tab $<$ $\circled{1}$ $>$\\
\\
$= gcd(a-k*b,b)$\\
\\
\tab $<$ setze $k = a\: div\: b$ $>$\\
\\
$= gcd(a-(a\: div\: b)*b,b)$\\
\\
\tab $<$ $\circled{2}$ $>$\\
\\
$= gcd(a\: mod\: b, b) =^{<symm.>} gcd(b, a\: mod\: b)$\\
\\
\\
\rule{\textwidth}{0.4mm}\\
\\

\subsection{Euklid schnell}

$\{a\geqslant 0 \wedge b\geqslant 0\}$\\
\\
$(x,y) := (a,b);$\\
\\
\uline{while} $y \neq 0$ \\
\\
\tab \uline{invariante} $gcd(a,b) = gcd(x,y) \wedge x\geqslant 0 \wedge y\geqslant 0$\\
\\
\tab \uline{do}\\
\\
\tab \tab$(x,y) = (y, x\: mod\: y)$\\
\\
\uline{endwhile}\\
\\
$\{x = gcd(a,b)\}$\\
\\
$\rightarrow$ rueckwerts einsetzen\\
\\
\\
Invar.$\circled{1}$ \tab\tab\tab\tab\tab\tab\tab Schleifenbedingung$\circled{4}$\\
\\
$\{ (gcd(a,b) = gcd(x,y))^{\circled{1}} \wedge (x\geqslant 0)^{\circled{2}} \wedge (y\geqslant 0)^{\circled{3}} \wedge (y\neq 0)^{\circled{4}}\}$\\
\\
VC\\
$\Rightarrow$\\
\\
$\{ (gcd(a,b) = gcd(y,x\: mod\: y))^{\circled{A}} \wedge (y\geqslant 0)^{\circled{B}} \wedge (x\: mod\: y \geqslant 0)^{\circled{C}} \}$ (PRE)\\
\\
\tab$(x,y) = (y, x\: mod\: y)$\\
\\
$\{ (gcd(a,b) = gcd(x,y)) \wedge (x\geqslant 0) \wedge (y\geqslant 0) \}$ (POST=inv)\\
\\
\\
Wir wissen: $\circled{1} \wedge \circled{2} \wedge\ \circled{3} \wedge \circled{4}$\\
Zu zeigen: $\circled{A}$, $\circled{B}$, $\circled{C}$\\
\\
$\circled{A}$: $gcd(a,b) =^{\circled{1}} gcd(x,y) =^{\circled{*}} gcd(y, x\: mod\: y)\: \checkmark$\\
\\
$\circled{B}$: $\circled{3} \Rightarrow \circled{B}\: \checkmark$\\
\\
$\circled{C}$: 'mod' immer $\geqslant 0$, und $y \neq 0$ wegen $\circled{4}$\\
\\
\\
\rule{\textwidth}{0.4mm}\\
\\

\subsection{Erweiterter Euklid}

\uline{Satz} Seien $a,b : \mathbb{Z}$ Dann gibt es Zahlen \\
\\
\tab $u,v: \mathbb{Z}$ mit\\
\\
\tab $u * a + v * b = gcd(a,b)$ (Bezout-Identitaet)\\
\\
\tab $\rightarrow$ \uline{nicht} eindeutig $\rightarrow$ unendlich\\
\\
$\qed$\\
\\
\\
Wir konstruieren $u$ und $v$ fuer $a,b : \mathbb{N}$ durch Erweiterten Euklid\\
\\
\\
$u*a + v*b = gcd(a,b)$\\
\\
$u' *a + v' *b =0$\\
\\
$(u+u')*a + (v+v') *b = gcd(a,b)$\\
\\
Waehle z.B. $u' =b$ und $v' = -a$\\
\\
\\
$\{ a\geqslant 0 \wedge b\geqslant 0\}$\\
\\
$(x,y) = (a,b)$\\
\\
$(u,u') = (1,0)$\\
\\
$(v,v') = (0,1)$\\
\\
$sign = +1$\\
\\
\\
\uline{while} $y \neq 0$ \\
\\
$I_1$\tab \uline{invariante} $gcd(x,y) = gcd(a,b) \wedge x\geqslant 0 \wedge y\geqslant 0$\\
$I_2$\tab \tab \tab $u*x + u'*y = a$\\
$I_3$\tab \tab \tab$v*x + v'*y = b$\\
$I_4$\tab \tab \tab$u*v' - u'*v = sign$\\
\\
\tab \tab($I_1, I_2, I_3, I_4$ unverknuepft)\\
\\
\uline{do}\\
\tab$(q,r) = (x\: div \:y, x\: mod\: y)$\\
\\
\tab$(x,y) = (y,r)$\\
\\
\tab$(u,u') = (q*u + u', u)$\\
\\
\tab$(v,v') = (q*v + v', v)$\\
\\
\tab$sign = -sign$\\
\\
\uline{endwhile}\\
\\
Post: $\{ x = gcd(a,b)\: \wedge $\\
\tab\tab $u*x = a\: \wedge$\\
\tab\tab $v*x = b\: \wedge$\\
\tab\tab $(sign *v')*a + (-sign*u')*b  = gcd(a,b) \}$\\
\\
\\
\rule{\textwidth}{0.4mm}\\
\\

\section{Dreizehnte Woche}

\subsection{GCD (Fortsetzung)}

$a,b : \mathbb{Z}$ Dann gibt es $u,v :\mathbb{Z}$ mit\\
\\
\tab $u*a + v*b  = gcd(a,b)$\\
\\
\uline{Zeigen Sie}\\
\\
$c\backslash a\:\: \circled{1}\tab   \wedge c\backslash b \:\: \circled{2}\tab  \equiv c\backslash gcd(a,b) $\\
\\
1) "$\Leftarrow$"\\
\\
Sei $d = gcd(a,b)$\\
\\
$c\backslash d$\\
\\
$d$ ist \uline{$gcd$} von $a$ und $b$. Also ist $d$ \uline{$cd$} (common divisor) von $a$ und $b$. Also $d\backslash a\wedge d\backslash b$\\
\\
$c\backslash d\wedge d\backslash a$, also $c\backslash a$\\
$c\backslash d\wedge d\backslash b$, also $c\backslash b$\\
\\
Transitivitaet $\backslash$\\
\\
2) "$\Rightarrow$"\\
\\
$c\backslash a\wedge c\backslash b$, also\\
\\
$\exists{k_1}:\mathbb{Z}$ mit $a=k_1*c$\\
$\exists{k_2}:\mathbb{Z}$ mit $b=k_2*c$\\
\\
Es gibt $u,v : \mathbb{Z}$ mit\\
\\
\tab $u*a + v*b = gcd(a,b)$\\
\\
also\\
\\
\tab $u*(k_1*c) + v*(k_2*c) = gcd(a,b)$\\
\\
also\\
\\
\tab $\underbrace{(u*k_1 + v*k_2)}_\text{$\mathbb{Z}$}*c = gcd(a,b)$\\
\\
also\\
\\
\tab $c\backslash gcd(a,b)$\\
\\
\\
\rule{\textwidth}{0.4mm}\\
\\

\subsection{Restklassen}

\uline{Def} (Menge der Reste modulo $n$)\\
\\
$\mathbb{Z}_n = \{ a:\mathbb{Z}\:\vert\: 0\leqslant a<n\}$\\
\\
\uline{Beispiel}\\
\\
$\mathbb{Z}_6 = \{0,1,2,3,4,5\}$\\
\\
\\
\uline{Arithmetik}\\
\\
$+_n , *_n : \mathbb{Z}_n \times \mathbb{Z}_n \rightarrow \mathbb{Z}_n$\\
\\
$a +_n b = (a+b)mod\:n$\\
$a *_n b = (a*b)mod\:n$\\
\\
\\
\rule{\textwidth}{0.4mm}\\
\\
\uline{Def} Sei $M$ eine Menge und $\circ$ eine Operation mit $\circ : M \times M \rightarrow M$. Das Paar $(M,\circ)$ heisst \uline{Gruppe}, wenn:\\
\\
\uline{1.} Es gibt ein $e:M$ mit \\
\tab $a\circ e = a= e\circ a$ fuer alle $a:M$ (Identitaete, neutr. Element)\\
\\
\uline{2.} Es gilt\\
\tab $(a\circ b)\circ c = a\circ(b\circ c)$ (Assoziativitaet)\\
\\
\uline{3.} Zu jedem $a:M$ gibt es ein $b:M$ mit \\
\tab $a\circ b = e = b\circ a$ (inverses Element) \tab ($e$: neutr. Elem. aus \uline{1.})\\
\\
\\
\uline{Beispiel} ($\mathbb{Z}_n, +_n$)\\
\\
\uline{1.} $0 +_n a = a = a +_n 0$\\
\\
\uline{Beweis}\\
\\
$0 +_n a$\\
\\
$= (0+a)\: mod\: a$\\
\\
$= a\: mod\: n$\\
\\
$= a\:\checkmark$\tab (weil $0\leqslant a<n$)\\
\\
\\
\uline{2.} $(a +_n b) +_n c = a +_n (b +_n c)$\\
\\
\uline{Bew.}\\
\\
$(a +_n b) +_n c$\\
\\
$= ((a+b)\: mod\: n + c)\: mod\: n$\\
$[a+b = n*q + (a+b) \: mod\: n] \in:\mathbb{Z}$\\
\\
$= ((a+b) - n*q +c)\: mod\: n$\\
\\
$= ((a+b)+c)\: mod\: n$\\
\\
$= (a+(b+c))\: mod\: n$\\
\\
$=$ das gleiche rueckwerts\\
\\
\\
\uline{3.} $a +_n ((-a)\: mod\: n)$\tab $\leftarrow$ inverses Element zu $a$\\
\\
$= (a+(-a) \: mod\: n)\: mod\: n$\\
\\
$=(a+ (-a))\: mod\: n$\\
\\
$= 0\: mod\: n$\\
\\
$= 0\: \checkmark$\\
\\
\\
\rule{\textwidth}{0.4mm}\\
\\
\uline{Beispiel} ($\mathbb{Z}_n, *_n$)\\
\\
1) $1 *_n a = a = a *_n 1$\\
\\
2) $(a *_n b) *_n c = a *_n (b *_n c)$\\
\\
3) im allgemeinen nicht erfuellt\\
\\
\\
\rule{\textwidth}{0.4mm}\\
\\
\uline{Def.} (\uline{reduzierte} Menge der Reste modulo $n$)\\
\\
\tab $\mathbb{Z}_{n}^{*} = \{ a:\mathbb{Z}\:\vert\: 0 \leqslant a< n \wedge gcd(a,n) = 1\}$\\
\\
\\
\uline{Beispiel}\tab $\mathbb{Z}_6 = \{0,1,2,3,4,5\}$ \\
\tab \tab $gcd(0,6)\:\:\:\:\:\: \circled{1}\: 2\: 3\:\: 2\: \circled{1}$\\
\\
\tab \tab $\mathbb{Z}_{6}^{+} = \{1,5\}$\\
\\
\\
\rule{\textwidth}{0.4mm}\\
\\
$gcd(a,1) = 1$\\
\\
Es gibt $u,v : \mathbb{Z}$ mit\\
\\
\tab $u*a + v*n =1$\\
\\
also\\
\\
\tab $(u*a + v*n)\: mod\: n =1 \: mod \: n$\\
\\
also\\
\\
\tab $(u*a)\: mod\: n =1$\\
\\
also\\
\\
\tab $u *_n a =1$\\
\\
Wir brauchen dazu noch ein Algorithmus (erweit. Euklid) ohne $v$ um Werte zu berechnen.\\
\\
\\
\uline{Beispiel:}\\
\\
$1 *_6 1 =1$\\
\\
$5 *_6 5 =1$\\
\\
Also: ($\mathbb{Z}_{n}^{\circled{*}\:!}, *_n$) ist eine Gruppe\\
\\
\\
Aufgabe: Tabelle fuer ($\mathbb{Z}_{15}^{*}, *_{15})$\\
\\
\\
\rule{\textwidth}{0.4mm}\\
\\

\section{Vierzehnte Woche}

\subsection{Restklassen Fortsetzung}

$+_n : \mathbb{Z}_n \times \mathbb{Z}_n \rightarrow \mathbb{Z}_n$\\
\\
$*_n$\\
\\
\\
$a +_n b = (a+b) \: mod\: n$\\
$a *_n b = (a * b)\: mod\: n$\\
\\
$a = b* q + r \wedge 0 \leqslant r < |b|$, \tab $q = a\: div\: b$, \tab $r = a\:mod\: b$\\
\\
\\
($\mathbb{Z}_n, +_n$) ist Gruppe\\
\\
($\mathbb{Z}_n, *_n$) \uline{keine} Gruppe\tab (inverses Element existiert nicht)\\
\\
$gcd(a,n) =1$\\
\\
$u*a + v*m =1$\\
\\
\\
\uline{Problem:} Gegeben $n:\mathbb{N}^{>0},\:a:\mathbb{Z}_{n}^{*}$\\
\\
\tab Finde $u = a^{-1} : \mathbb{Z}_{n}^{*} $ mit\\
\\
\tab $u *_n a =1$\\
\\
\\
$\{n\geqslant 0 \wedge a\geqslant 0 \wedge gcd(a,n) = 1\}$\\
\tab $(x,y) := (n,a)$\\
\tab $(u,u') := (1,0)$\\
\tab $sign := +1$\\
\\
\uline{while} $y \neq 0$\\
\\
\tab \uline{invar} $gcd(x,y) = 1 \wedge x\geqslant 0 \wedge y\geqslant 0$\\
\tab \tab $u*x + u' * y =n$\\
\\
\uline{do}\\
\\
\tab $(q,r) := (x \: div\: y, x\: mod\: y)$\\
\tab $(x,y) := (y,r)$\\
\tab $(u,u') := (q*u + u', u)$\\
\tab $sign = -sign$\\
\\
\uline{endwhile}\\
\\
$\{ x =1 \wedge u = n\} \leftarrow$ check!\\
\\
$inverse := (-sign * u')\: mod\: n$\\
\\
$\{inverse *_n a =1\}$\\
\rule{\textwidth}{0.4mm}\\



%\subsection{This is a subsection}

%
% TODO: This is a comment
% COMMENT
%


\end{document}