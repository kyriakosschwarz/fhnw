\documentclass[18pt,a4paper]{article}
\usepackage[utf8]{inputenc}
\usepackage{amsmath}
\usepackage{amsfonts}
\usepackage{amssymb}
\author{platzh1rsch}
\title{Diskrete Mathematik 1}
\begin{document}

\section{Quantifizierung}

$\sum_{i=1}^{n}i^2 = 1^2 + 2^2 + ... + n^2 $\\
\\
$\sum_{i=1}^{-1}i^2 = 0 $ (neutrales Element)\\
\\
$\sum_{i=1}^{n}i^2 + 1 = 1^2 + 2^2 + ... + n^2 + 1 $ (?)\\
\\
$\sum_{i=1}^{n}(i^2 + 1) = (1^2 + 1) + (2^2 + 1) + ... + (n^2 + 1) $ (?)\\
\\
$\sum_{\substack{i=1\\odd(i)}}^{n}i^2 = 1^2 + 3^2 + ... + n^2 $\\
\\
\\
$\prod_{i=1}^{n}i^2 = 1^2 * 2^2 * ... * n^2 $\\
\\
$\prod_{i=1}^{-1}i^2 = 1 $ (neutrales Element)\\
\\
\\
$\forall_{i=0}^{n-1} b[i] == 0 = b[0]==0 \wedge b[1]==0 \wedge ... \wedge b[n-1]==0 $\\
\\
$\exists_{i=0}^{n-1} b[i] == 0 = b[0]==0 \vee b[1]==0 \vee ... \vee b[n-1]==0 $\\
\\
\\
$\sum_{i=1}^{n}i^2 = ( \sum{i} : \mathbb{N}  \:\vert\:  1\leqslant i \leqslant n : i^2 ) $\\
\\
$                    ( \sum{i} : \mathbb{N}, j:\mathbb{N}  \:\vert\:  1\leqslant i \leqslant 2 \wedge 1\leqslant j\leqslant 3 : i^j ) $\\
\\
\\
\\
$( \circ v_1 : T_1, ... , v_n : T_n \:\vert\: R : P ) $\\
\\
- $ \circ : T \times T \rightarrow T $ (wobei T ein Typ ist)\\
\\
Bsp: $+ : \mathbb{N} \times \mathbb{N} \rightarrow \mathbb{N}$\\
$+ (3,4) = 7$\\
\\
\\
$a \circ b = b \circ a$ fuer alle a,b : T (Symmetrie) --- ABELSCHES MONOID\\
$(a \circ b) \circ c = a \circ (b \circ c)$ fuer alle a,b,c : T (Assoziativitaet)\\
$u \circ a = a = a \circ u$\\
es gibt ein u : T, so dass fuer alle a : T (neutrales Element) --- MONOID\\
\\
\\
\begin{tabular}{ l c r }
  $\circ$ & $T$ & $u$ \\
  $+\sum$ & $\mathbb{Z}$ & $0$ \\
  $*\prod$ & $\mathbb{Z}$ & $1$ \\
  $\forall$ & $\mathbb{B}$ & $true$\\
  $\exists$ & $\mathbb{B}$ & $false$
\end{tabular}
\\
\\
String mit Konkatenation nicht-abelsches Monoid\\
$("a" + "b") + "c" \:equals\: "a" + ("b" + "c")$\\
$"a" + "" \:equals\: "a"$\\
$"a" + "b" \:!equals\: "b" + "a"$\\
\\
- $T_1, ... , T_n$ Datentypen\\
\\
- $V_1, ... , V_n$ Variablen\\
\\
alle paarweise verschieden\\
\\
$V_i \:vom\: Typ\: T_i$\\
\\
- R : boolescher Ausdruck, kann $V_1 ... V_n$ enthalten, Bereich (Range)\\
\\
- P : beliebiger Ausdruck vom Typ T, kann $V_1 ... V_n$ enthalten, Koerper (Body)\\
\\
Typ der Quantifizierung : T\\
\\
\\
$(\forall i : \mathbb{N} \:\vert\: 0\leqslant i \leqslant n : b[i]=0)$ und das Ganze ist : $\mathbb{B}$\\
$(\circ\: V_1 : T_1 \:\vert\: R : P )$ wobei $T_1 : \mathbb{N}, P : \mathbb{B}$\\  
\\
$\wedge : \mathbb{B} \times \mathbb{B} \rightarrow \mathbb{B}$\\
\\
$P : T_1 \times T_2 \times ... \times T_n \rightarrow T$\\
\\
\\
\section{Semantik}

Bsp: $(+ i:\mathbb{Z} \:\vert\: -1\leqslant i\leqslant 2 : i^2)$\\
\\
\\
1. Fall (Topf $\neq \emptyset$)\\
\\
Von $\mathbb{Z}$ alle Zahlen ausfiltern (-1,0,1,2) (Menge)\\
\\
$\rightarrow^{1^2} ((-1)^2, 1^2, 0^2, 2^2) (1,1,0,4)$ (Multimenge)\\
\\
$\rightarrow 2^2 + 1^2 + (-1)^2 + 0^2$\\
\\
\\
2. Fall (Topf = 0)\\
\\
$\rightarrow$ Topf leer $\rightarrow$ Resultat: Neutrales Element (von +) $\rightarrow$ 0\\
\\
\\
Beispiele:\\
\\
1) $(+ \:i:\mathbb{N} \:\vert\: 0\leqslant i\leqslant 4 : i*8) = (0*8) + (1*8) + ... $\\
\\
2) $(* \:i:\mathbb{N} \:\vert\: 0\leqslant i\leqslant 3 : i+1) = (0+1) * (1+1) * ... $\\
\\
3) $(\wedge \:i:\mathbb{N} \:\vert\: 0\leqslant i\leqslant 2 : i*d \neq 6) = (0*d) \neq 6 \wedge (1*d) \neq 6 \wedge ...  $\\
\\
4) $(\vee \:i:\mathbb{N} \:\vert\: 0\leqslant i\leqslant 21 : b[i]=0) = b[0] = 0 \vee b[1] = 0 \vee ...  $\\
\\
5) $(\sum k : \mathbb{N} \:\vert\: k^2 = 4 : k^2) = 2$\\
\\
6) $(\sum k : \mathbb{Z} \:\vert\: k^2 = 4 : k^2) = 2 + (-2) = 0$\\
\\
\\







\subsection{This is a subsection}

%
% TODO: This is a comment
% COMMENT
%


\end{document}