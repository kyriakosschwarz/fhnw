\documentclass[18pt,a4paper]{article}
\usepackage[utf8]{inputenc}
\usepackage{amsmath}
\usepackage{amsfonts}
\usepackage{amssymb}
\usepackage{tikz}
\newcommand*\circled[1]{\tikz[baseline=(char.base)]{
            \node[shape=circle,draw,inner sep=1pt] (char) {#1};}}
%\author{kyriakosschwarz}
\title{Diskrete Mathematik 2}



%Math
\usepackage{amsmath}
\usepackage{amsfonts}
\usepackage{amssymb}
\usepackage{amsthm}
\usepackage{ulem}

%PageStyle
%\usepackage[ngerman]{babel} % deutsche Silbentrennung
\usepackage[utf8]{inputenc} 
\usepackage{fancyhdr, graphicx}
\usepackage[scaled=0.92]{helvet}
\usepackage{enumitem}
\usepackage{parskip}
\usepackage[a4paper,top=2cm]{geometry}
\setlength{\textwidth}{17cm}
\setlength{\oddsidemargin}{-0.5cm}


% Shortcommands
\newcommand{\Bold}[1]{\textbf{#1}} %Boldface
\newcommand{\Kursiv}[1]{\textit{#1}} %Italic
\newcommand{\T}[1]{\text{#1}} %Textmode
\newcommand{\Nicht}[1]{\T{\sout{$ #1 $}}} %Streicht Shit durch

%Arrows
\newcommand{\lra}{\leftrightarrow} 
\newcommand{\ra}{\rightarrow}
\newcommand{\la}{\leftarrow}
\newcommand{\lral}{\longleftrightarrow}
\newcommand{\ral}{\longrightarrow}
\newcommand{\lal}{\longleftarrow}
\newcommand{\Lra}{\Leftrightarrow}
\newcommand{\Ra}{\Rightarrow}
\newcommand{\La}{\Leftarrow}
\newcommand{\Lral}{\Longleftrightarrow}
\newcommand{\Ral}{\Longrightarrow}
\newcommand{\Lal}{\Longleftarrow}

%Mine(new)
\newcommand{\tab}{\hspace*{2em}}


%Metadata
%\fancyfoot[C]{If you use this documentation for a exam, you should offer a beer to the authors!}
\title{
	\vspace{5cm}
	Diskrete Mathematik 2 \\
}
%\author{Kyriakos Schwarz}
\date{FS 2013}





\begin{document}

% Titelbild
\maketitle
\thispagestyle{fancy}
\newpage

% Inhaltsverzeichnis
%\pagenumbering{Roman}
\tableofcontents	  	
\newpage

%\setcounter{page}{1}
\pagenumbering{arabic}

% Inhalt Start

\section{Erste Woche}


\subsection{Semesterablauf}

- Arithmetik in $\mathbb{Z}$\\
\\
- Modulares Rechnen\\
\\
- Gruppen\\
\\
- RSA\\
\\
- Quantifizierung\\
\\
- Induktion\\
\tab - Rekurision\\
\tab - Invarianten\\
\\
\rule{\textwidth}{0.4mm}\\
\\
- Kein Laptop\\
\\
- Zwischenpruefung: 30.04.2013 (1 Stunde)\\
\\
- 5. Maerz 2013 Unterricht nur bis 18:20\\
\\
\\
\rule{\textwidth}{0.4mm}\\
\\
- Buecher:\\
\\
\tab - Gries/Schneider\\
\tab \tab A logical approach to Discrete Math\\
\tab \tab Springer, 1993\\
\\
\tab - Jean Gallier\\
\tab \tab Discrete Math\\
\tab \tab Springer, 2010\\
\\
\tab - Struckermann/Waetiger\\
\tab \tab Mathematik fuer Informatiker\\
\tab \tab Spektrum, 2007\\
\\
\\
\rule{\textwidth}{0.4mm}\\
\\



\subsection{Quantifizierung}

$\mathbb{N} = \begin{cases} 
		 \{\underline{0},1,2,...\} \:(?)\\
		 \{1,2,...\} \:(?)\\ 
		 \end{cases}$ 


$\sum_{i=1}^{n}i^2 = 1^2 + 2^2 + ... + n^2 $\\
\\
$\sum_{i=1}^{-1}i^2 = \begin{cases}
                       ungueltig \:(?)\\
                       1^2 \:(?)\\
                       1^2 + 0^2 + (-1)^2 \:(?)\\
                       0 \:(\ra ja, \:Neutrales \:Element)\\
                      \end{cases}$\\                           
\\
$\sum_{i=1}^{n}i^2 + 1 = 1^2 + 2^2 + ... + n^2 + 1 $ (?)\\
\\
$\sum_{i=1}^{n}(i^2 + 1) = (1^2 + 1) + (2^2 + 1) + ... + (n^2 + 1) $ (?)\\
\\
$\sum_{\substack{i=1\\odd(i)}}^{n}i^2 = 1^2 + 3^2 + ... + n^2 $ , falls $odd(n)$, sonst $(n-1)^2$\\ 
\\
\rule{\textwidth}{0.4mm}\\
\\
$\prod_{i=1}^{n}i^2 = 1^2 * 2^2 * ... * n^2 $\\
\\
$\prod_{i=1}^{-1}i^2 = 1 $ (neutrales Element)\\
\\
\\
\rule{\textwidth}{0.4mm}\\
\\
(Java ==)\\
\\
$\forall_{i=0}^{n-1} (b[i] == 0) = (b[0]==0) \wedge (b[1]==0) \wedge ... \wedge (b[n-1]==0) $\\
\\
$\exists_{i=0}^{n-1} (b[i] == 0) = (b[0]==0) \vee (b[1]==0) \vee ... \vee (b[n-1]==0) $\\
\\
\\
\rule{\textwidth}{0.4mm}\\
\\
$\sum_{i=1}^{n}i^2 = ( \sum{i} : \mathbb{N}  \:\vert\:  1\leqslant i \leqslant n : i^2 ) $\\
\\
\tab \tab $( \sum{i} : \mathbb{N}, j:\mathbb{N}  \:\vert\:  1\leqslant i \leqslant 2 \wedge 1\leqslant j\leqslant 3 : i^j ) $\\
\\
\\
\rule{\textwidth}{0.4mm}\\
\\
$( \circ \:v_1 : T_1, ... , v_n : T_n \:\vert\: R : P ) $\\
\\
- $ \circ : T \times T \rightarrow T $ (wobei T ein Typ ist)\\
\\
Bsp: $+ : \mathbb{N} \times \mathbb{N} \rightarrow \mathbb{N}$\\
$+ (3,4) = 7$\\
\\
\\
ABELSCHES MONOID$\begin{cases} 
  a \circ b = b \circ a $ fuer alle $ a,b : T $ (Symmetrie)(Kommutativitaet)$\\
  MONOID\begin{cases}
    (a \circ b) \circ c = a \circ (b \circ c) $ fuer alle $ a,b,c : T $ (Assoziativitaet)$\\
    u \circ a = a = a \circ u\\
    $ es gibt ein $u : T$, so dass fuer alle $ a : T $ (neutrales Element)$\\
  \end{cases}
\end{cases}$\\
\\
\\
\rule{\textwidth}{0.4mm}\\
\\
\begin{tabular}{ l c r }
  $\circ$ & $T$ & $u$ \\
  $+\sum$ & $\mathbb{Z}$ & $0$ \\
  $*\prod$ & $\mathbb{Z}$ & $1$ \\
  $\forall$ & $\mathbb{B}$ & $true$\\
  $\exists$ & $\mathbb{B}$ & $false$
\end{tabular}
\\
\\
\rule{\textwidth}{0.4mm}\\
\\
String mit Konkatenation: Nicht-abelsches Monoid\\
$("a" + "b") + "c" \:equals\: "a" + ("b" + "c")$\\
$"a" + "" \:equals\: "a"$\\
$"a" + "b" \:!equals\: "b" + "a"$ (nicht equals)\\
\\
- $T_1, ... , T_n$ Datentypen\\
\\
- $V_1, ... , V_n$ Variablen\\
\\
alle paarweise verschieden\\
\\
$V_i$ vom Typ: $T_i$\\
\\
\\
- R : boolescher Ausdruck, kann $V_1 ... V_n$ enthalten, Bereich (Range)\\
\\
- P : beliebiger Ausdruck vom Typ T, kann $V_1 ... V_n$ enthalten, Koerper (Body)\\
\\
Typ der Quantifizierung : T\\
\\
\\
\rule{\textwidth}{0.4mm}\\
\\
$(\forall i : \mathbb{N} \:\vert\: 0\leqslant i \leqslant n : b[i]=0)$ und das Ganze ist : $\mathbb{B}$\\
$(\circ\: V_1 : T_1 \:\vert\: R : P )$ wobei $T_1 : \mathbb{N}, P : \mathbb{B}$\\  
\\
$\wedge : \mathbb{B} \times \mathbb{B} \rightarrow \mathbb{B}$\\
\\
$P : T_1 \times T_2 \times ... \times T_n \rightarrow T$\\
\\
\\
\rule{\textwidth}{0.4mm}\\
\\




\subsection{Semantik}

Bsp: $(+ i:\mathbb{Z} \:\vert\: -1\leqslant i\leqslant 2 : i^2)$\\
\\
\\
1. Fall (Topf $\neq \emptyset$)\\
\\
Von $\mathbb{Z}$ alle Zahlen ausfiltern (-1,0,1,2) (Menge)\\
\\
$\rightarrow^{1^2} ((-1)^2, 1^2, 0^2, 2^2) (1,1,0,4)$ (Multimenge)\\
\\
$\rightarrow 2^2 + 1^2 + (-1)^2 + 0^2$\\
\\
\\
2. Fall (Topf = 0)\\
\\
$\rightarrow$ Topf leer $\rightarrow$ Resultat: Neutrales Element (von +) $\rightarrow$ 0\\
\\
\\
\rule{\textwidth}{0.4mm}\\
\\
Beispiele:\\
\\
1) $(+ \:i:\mathbb{N} \:\vert\: 0\leqslant i < 4 : i*8) = (0*8) + (1*8) + ... $\\
\\
2) $(* \:i:\mathbb{N} \:\vert\: 0\leqslant i < 3 : i+1) = (0+1) * (1+1) * ... $\\
\\
3) $(\wedge \:i:\mathbb{N} \:\vert\: 0\leqslant i < 2 : i*d \neq 6) = ((0*d) \neq 6) \wedge ((1*d) \neq 6) \wedge ...  $\\
\\
4) $(\vee \:i:\mathbb{N} \:\vert\: 0\leqslant i < 21 : b[i]=0) = (b[0] == 0) \vee (b[1] == 0) \vee ...  $\\
\\
5) $(\sum k : \mathbb{N} \:\vert\: k^2 = 4 : k^2) = 2$\\
\\
6) $(\sum k : \mathbb{Z} \:\vert\: k^2 = 4 : k^2) = 2 + (-2) = 0$\\
\\
\\
\\
\rule{\textwidth}{0.4mm}\\
\\


\section{Zweite Woche}

\subsection{Freie/Gebundene Variablen}

$( \circ \: v_1 : T_1,...,v_n : T_n \:\vert\: R : P )$\\
\\
\\
\rule{\textwidth}{0.4mm}\\
\\
E1: \tab$(\sum{i} : \mathbb{Z} \:\vert\: 0\leqslant i < n : i^2 )$\\
\\
- Wert haengt von $n$ ab, nicht von $i$\\
\tab $n = 3: \tab0\tab1\tab2$\\
\\
\tab\tab\tab $0^2 + 1^2 + 2^2 = 5$\\
\\
\tab $n = 0:$ kein $i$\\
\\
\tab\tab $0$ (neutral $+$)\\
\\
\\
\rule{\textwidth}{0.4mm}\\
\\

E2: \tab$(\sum{j} : \mathbb{Z} \:\vert\: 0\leqslant j < n : j^2 )$\\
\\
$n = 3 \rightarrow 5$\\
\\
$n = 0 \rightarrow 0$\\
\\
\\
\rule{\textwidth}{0.4mm}\\
\\
E3: $(\sum{i} \:\circled{1} : \mathbb{Z} \:\vert\: 0\leqslant i \:\circled{2} < n : i^2 \:\circled{3} ) + 1 \:\circled{4}$\\
\\
$(\leftarrow \rightarrow) :$ Gueltigkeitsbereich von $i$ (scope)\\
\\
$i$ tritt hier 4 mal auf (occurs)\\
\\
Auftreten (occurances) $\circled{1}, \circled{2}, \circled{3}$ gebunden\\
Auftreten $\circled{4}$ \uline{frei}\\
$\circled{2}$ und $\circled{3}$ gebunden an $\circled{1}$\\
$\circled{2}$ und $\circled{3}$ angewandte Auftreten (applied)\\
$\circled{1}$ bindende, deklarierende Auftreten (binding)\\
\\
\\
Eine Variable heisst \uline{frei} in einem Ausdruck E (expresion), falls sie in E frei vorkommt.\\
\\
\\
$FV(E) =$ Menge der freie Variablen von E\\
\\
$FV(E_3) = \{'n', 'i'\}$ (Die Variablennamen und nicht die Werte der Variablen)\\
\\
\\
\rule{\textwidth}{0.4mm}\\
\\
$x, y : \mathbb{Z}$\\
$x = 3, y = 5$\\
$\{x, y\} = \{3, 5\}$\\
\\
$x = y = 3$\\
$\{x, y\} = \{3\}$\\
\\
$x + y * 2$\\
$y , 2 : *$ Operator\\
dann das Resultat mit $x$ und $+$ Operator\\
\\
\\
\rule{\textwidth}{0.4mm}\\
\\
E4: $(\sum{i} : \mathbb{Z} \:\vert\: 0\leqslant i< n : i^2 ) * (\sum{i} : \mathbb{Z} \:\vert\: 0\leqslant i< n : i^3 )$\\
\\
$FV(E_4) = \{'n'\}$\\
\\
\\
\rule{\textwidth}{0.4mm}\\
\\
E5: $( \prod{n} \:\vert\: k\leqslant n \leqslant l : (\sum{i} : \mathbb{Z} \:\vert\: 0\leqslant i< n : i^2 ) * (\sum{i} : \mathbb{Z} \:\vert\: 0\leqslant i< n : i^3 ) )$\\
\\
$FV(E_5) = \{'k', 'l'\}$\\
\\
\\
\rule{\textwidth}{0.4mm}\\
\\
E6: $(\sum{i} : \mathbb{Z} \:\vert\: 0\leqslant i\leqslant (\sum{i} : \mathbb{Z} \:\vert\: 2\leqslant i< 3 : i^2 ) : i^2 )$\\
\\
$FV(E_6) = \emptyset$\\
\\
Ein Ausdruck E ohne freie Variablen ($FV(E) = \emptyset$ oder $\{\}$) heisst geschlossen\\
\\
\\
\rule{\textwidth}{0.4mm}\\
\\
 $(\sum{i} : \mathbb{Z} \:\vert\: 1\leqslant i< 2 :  (\sum{j} : \mathbb{Z} \:\vert\: 1\leqslant j< 3 : i + j ) )$\\
\\
\uline{$i$ zuerst:}\\
\\
$i :$ \tab\tab\tab1 \tab\tab\tab\tab\tab\tab2\\
\tab $(\sum{j} : \mathbb{Z} \:\vert\: 1\leqslant j< 3 : 1 + j ) + (\sum{j} : \mathbb{Z} \:\vert\: 1\leqslant j< 3 : 2 + j )$\\
\\
$j :$ \tab\:1 \tab\:\:\:\:\:\:2 \tab\:\:\:\:\:\:3\tab\tab\:1 \tab\:\:\:\:\:\:\:2 \tab\:\:\:\:\:3\\
\tab$((1+1)+(1+2)+(1+3)) + ((2+1)+(2+2)+(2+3))$
\\
\\
\uline{$j$ zuerst:}\\
\\
$j :$ \tab\tab\tab\tab\tab\:\:\:\:\:1 \tab\:\:\:\:\:\:2 \:\:\:\:\:\tab3\\
\tab$(\sum{i} : \mathbb{Z} \:\vert\: 1\leqslant i< 2 :  ((i+1)+(i+2)+(i+3)) )$\\
\\
$i :$ \tab\tab\tab\:\:\:1 \tab\tab\tab\tab\tab\tab\:\:\:2\\
\tab$((1+1)+(1+2)+(1+3)) + ((2+1)+(2+2)+(2+3))$\\
\\
\\
\rule{\textwidth}{0.4mm}\\
\\
\\

\section{Dritte Woche}

\subsection{Saetze zur Quantifizierung}

\uline{Satz} (Dummy renaming)\\
\\
$(\circ \:v \:\vert\: R : P ) = (\circ \:w \:\vert\: R[v\leftarrow w] : P[v\leftarrow w] )$\\
\\
\uline{Voraussetzung:} $w \:\notin FV(R)\cup FV(P)$\\
\\
\uline{Dabei:} $E[v\leftarrow F]$ bezeichnet exakt denselben Ausdruck wie $E$, aber alle \uline{freien} Auftreten von $v$ ersetzt durch $(F)$.\\
\\
wobei $E, F:$ Ausdruck, $v:$ Variable\\
\\
\\
\rule{\textwidth}{0.4mm}\\
\\
Bsp: $(i+5)[i\leftarrow j+3] = (j+3)+5$\\
\\
wobei $(i+5): E$, $[i: v$, $j+3: F]$\\
\\
\\
$(i*5)[i\leftarrow j+3] = (j+3)*5$\\
\\
\\
$(\sum{i} \:\vert\: true : i^2)[i\leftarrow j+3] = (\sum{i} \:\vert\: true : i^2)$\\
\\
$(\sum{i} \:\vert\: true : i^2) = (\sum{j} \:\vert\: true : j^2)$\\
\\
$= (\sum{j} \:\vert\: true[i\leftarrow j] : i^2[i\leftarrow j])$\\
\\
$= (\sum{j} \:\vert\: true : j^2)$\\
\\
$42[i\leftarrow j+3] = 42$ "Man kann die Bedeutung des Universums nicht aendern."\\
\\
\\
\rule{\textwidth}{0.4mm}\\
\\
Es ist ein Unterschied, ob die Ersetzung innerhalb oder ausserhalb einer Quantifizierung angegeben wird.\\
\\
$(\sum{i} \:\vert\: true : i^2)[i \leftarrow j + 3]$\\
\\
Hier sollen alle freien Auftreten von Variable $i$ in $(\sum{i} \:\vert\: true : i^2)$ durch $j + 3$ ersetzt werden.\\
Aber alle Auftreten von $i$ sind in diesem Ausdruck gebunden, also ist nichts zu ersetzen.\\
\\
Dummy renaming sagt aus, dass wir die gebundenen Auftreten einer Variablen innerhalb einer Quantifizierung\\
konsistent umbenennen duerfen, solange wir dabei keine freien Variablen einfangen.\\
\\
$(\sum{i} \:\vert\: 1\leqslant i \leqslant n : i^2)$\\
$= (\sum{j} \:\vert\: 1\leqslant i \leqslant n[i \leftarrow j]: i^2[i \leftarrow j])$\\
$= (\sum{j} \:\vert\: 1\leqslant j \leqslant n : j^2)$\\
\\
Hier sind die Ersetzungen innerhalb der Quantifizierung.\\  
Und beachten Sie: im Teilausdruck $1\leqslant i\leqslant n$ ist die Variable $i$ frei,\\
daher liefert $1\leqslant i\leqslant n[i \leftarrow j]$ den Ausdruck $1\leqslant j\leqslant n$\\
\\
Im Gesamtausdruck $(\sum{i} \:\vert\: true : i^2)$ sind alle Auftreten von $i$ hingegen gebunden.\\
Aber in diesem Ausdruck wollen wir auch nicht ersetzen, sondern eben in den beiden Teilausdruecken.\\
\\
Ein Auftreten einer Variablen kann in einem Teilausdruck frei sein, aber im Gesamtausdruck gebunden.\\
Ob ein Auftreten frei oder gebunden ist, hängt immer vom betrachteten (Teil-)Ausdruck ab.\\
\\
\\
\rule{\textwidth}{0.4mm}\\
\\
Bsp: $(\sum{i} \:\vert\: 1\leqslant i\leqslant n : i^2 )$\\
\\
wobei $i: v$, $(1\leqslant i\leqslant n): R$, $i^2: P$\\
\\
$= (\sum{j} \:\vert\: (1\leqslant i\leqslant n)[i\leftarrow j] : i^2[i\leftarrow j] )$\\
\\
wobei $j: w$\\
\\
$= (\sum{j} \:\vert\: 1\leqslant j\leqslant n : j^2)$\\
\\
\\
\uline{Aber:} Vorsicht:\\
\\
$(\sum{i} : \:\vert\: 1\leqslant i\leqslant n : i^2 )$\\
$n=0 , \tab 0(neutral +)$\\
$n=1 , \tab 1$\\
\\
haengt von $n$ ab\\
\\
$\neq$\\
\\
$(\sum{n} : \:\vert\: 1\leqslant n\leqslant n : n^2 )$\\
$\infty$ undefiniert\\
\\
haengt \uline{nicht} von $n$ ab\\
\\
\\
\rule{\textwidth}{0.4mm}\\
\\

\section{Vierte Woche}

\subsection{Saetze zur Quantifizierung (Fortsetzung)}

$(\sum{i} \:\vert\: 0\leqslant i < n : i^2)[n\leftarrow n^2] = (\sum{i} \:\vert\: 0\leqslant i < n^2 : i^2)$\\
\\
$(\sum{i} \:\vert\: 0\leqslant i < n : i^2)[n\leftarrow i+1] \neq (\sum{i} \:\vert\: 0\leqslant i < i+1 : i^2)$ (geht nicht)\\
freies Auftreten von $i$ wird gefangen $\rightarrow$ name clash\\
\\
\\
$(\sum{i} \:\vert\: 0\leqslant i < n : i^2)[n\leftarrow i+1]$\\
$= (\sum{j} \:\vert\: 0\leqslant j < n : j^2)[n\leftarrow i+1]$\\
$= (\sum{j} \:\vert\: 0\leqslant j < i+1 : j^2)$\\
\\
\\
\rule{\textwidth}{0.4mm}\\
\\
\uline{Empty range}\\
\\
$(\circ \:v \:\vert\: false : P ) = u_\circ$ (Neutrales Element)\\
\\
\\
\rule{\textwidth}{0.4mm}\\
\\
\uline{One point}\\
\\
Voraussetzung: $v \:\notin FV(E)$\\
\\
$(\circ \:v \:\vert\: v = E : P ) = P[v\leftarrow E]$\\
\\
Bsp. $(\sum{i} \:\vert\: i = j + 3 : i^2) = i^2[i\leftarrow j+3] = (j+3)^2$\\
\\
\tab $(\sum{i} \:\vert\: i = j + i + 3 : i^2) \neq i^2[i\leftarrow j+i+3] = (j+i+3)^2$ (geht nicht)\\
\\
\\
\rule{\textwidth}{0.4mm}\\
\\
\uline{Split-off term}\\
\\
$(\circ \:i \:\vert\: 0\leqslant i < n+1 : P) = (\circ \:i \:\vert\: 0\leqslant i < n : P) \circ P[i\leftarrow n]$\\
\\
Bsp. $(\sum{i} \:\vert\: 0\leqslant i < n+1 : i^2) \tab = (\sum{i} \:\vert\: 0\leqslant i < n : i^2) + n^2$\\
\tab $0^2 + 1^2 + ... + (n-1)^2 + n^2 \:\:\:= (0^2 + 1^2 + ... + (n-1)^2) + n^2$\\
\\
$n=0 :$\\
\tab $(\circ \:i \:\vert\: 0\leqslant i < 1 : P) = (\circ \:i \:\vert\: 0\leqslant i < 0 : P) \circ P[i\leftarrow 0]$\\
\tab $i=0 :$\\
\tab \tab $P[i\leftarrow 0]$ (One point) $= u_\circ $(empty range)$  \:\circ P[i\leftarrow 0]$\\
\\
\\
\rule{\textwidth}{0.4mm}\\
\\

\subsection{Anwendung}

\uline{Praedikat}\\
\\
$i+1 > j : Bool$ macht Aussage ueber Werte von freien Variablen\\
\\
Feld $b[0...n-1]$ mit ganzen Zahlen$; n\geqslant 0$\\
\\
\\
\rule{\textwidth}{0.4mm}\\
\\
"$b$ enthaelt eine $-1$." $\rightarrow$ bedeutet mindestens\\
\\
$(\exists{i} : \mathbb{N} \:\vert\: 0\leqslant i < n : b[i] = -1)$\\
\\
\\
\rule{\textwidth}{0.4mm}\\
\\
"$b$ enthaelt genau eine $-1$."\\
\\
$(\exists{i} : \mathbb{N} \:\vert\: 0\leqslant i < n : (b[i] = -1) \wedge (\forall{j} : \mathbb{N} \:\vert\: (0\leqslant j < n) \wedge (j\neq i) : b[j] \neq -1 ))$\\
\\
$=$\\
\\
$1 = (\sum{i} : \mathbb{N} \:\vert\: (0\leqslant i < n) \wedge (b[i] = -1 : 1)$\\
\tab \tab \tab \tab \tab \:\:\:$\&\&$\\
\\
\\
\rule{\textwidth}{0.4mm}\\
\\
"$b$ enthaelt keine $-1$."\\
\\
$(\forall{i} : \mathbb{N} \:\vert\: 0\leqslant i < n : b[i] \neq -1)$\\
\\
$=$\\
\\
$\neg (\exists{i} : \mathbb{N} \:\vert\: 0\leqslant i < n : b[i] = -1) \rightarrow$ ($\neg$ ("$b$ enthaelt mindestens eine $-1$.")) \\
\\
\\
\rule{\textwidth}{0.4mm}\\
\\
$\neg (\exists{v} \:\vert\: R : P) = (\forall{v} \:\vert\: R : \neg P)$\\
$\neg (\forall{v} \:\vert\: R : P) = (\exists{v} \:\vert\: R : \neg P)$\\
\\
\\
\\
de Morgan\\
$\neg (\exists{v} \:\vert\: R : P) = \neg (P_0 \vee P_1 \vee ... \vee P_{n-1} \vee P_n) $\\
\tab \tab \tab $= ((\neg P_0) \wedge (\neg P_1) \wedge ... (\neg P_n))$\\
\tab \tab \tab $= (\forall{v} \:\vert\: R : \neg P)$\\
\\
\\
\rule{\textwidth}{0.4mm}\\
\\

\section{Fuenfte Woche}

\subsection{Magisches Quadrat}

\uline{Uebungsblatt 2, Aufgabe 3}\\
\\
$k,i: 1\leqslant k\leqslant n, 1\leqslant i\leqslant n$\\
\\
$(\exists{M} : N \:\vert\: true : (\forall{i} \:\vert\: 1\leqslant i\leqslant n :(\sum{k} \:\vert\: 1\leqslant k\leqslant n : Q[i,k]) = M $\\
\tab\tab\tab\tab\tab\tab\tab $\wedge (\sum{k} \:\vert\: 1\leqslant k\leqslant n : Q[k,i]) = M$ \\
\tab\tab\tab\tab\tab\tab\tab $\wedge (\sum{k} \:\vert\: 1\leqslant k\leqslant n : Q[k,k]) = M$ \\
\tab\tab\tab\tab\tab\tab\tab $\wedge (\sum{k} \:\vert\: 1\leqslant k\leqslant n : Q[k,(n+1)-k)]) = M$\\
\tab\tab\tab\tab\tab\tab\tab $\wedge (\forall{m} : N \:\vert\: 1\leqslant m\leqslant n^2 :$\\
\tab\tab\tab\tab\tab\tab\tab\tab $(\exists{i,j} \:\vert\: 1\leqslant i < n\wedge 1\leqslant j < n: m = Q[i,j])))$ \\
\\
\\
\rule{\textwidth}{0.4mm}\\
\\

\subsection{Aequivalenz}

Math symbols????????????????????????????????????????????????????????\\

\subsection{Mathematische Induktion}

$(\mathbb{B}: Boolean)$\\
\\
Sei $P : \mathbb{N} \rightarrow \mathbb{B}$\\
zu zeigen:\\
$(\forall{n} : \mathbb{N} \:\vert\: true : P(n) )$\\
\\
\uline{Beispiel}\\
\\
$P(n): n^3 + 5*n$ ist ein Vielfaches von $6$\\
\\
$z$ ist Vielfaches von $6$ heisst:\\
$(\exists{i} : \mathbb{Z} \:\vert\: true : i*6  = z)$\\
\\
$0^3 + 5*n = 0$(Zeuge) $*\:6$\\
$1^3 + 5*n = 1*6$\\
$2^3 + 5*n = 2*6$  (Muss bei allen $true$ zurueck geben!!)\\
\\
\\
\uline{Idee}: \uline{Induktionsprinzip}\\
\\
Man zeigt:\\
\\
1) $P(0)$\\
\\
2) $P(n) \Rightarrow P(n+1)$ fuer alle $n : \mathbb{N}$\\
\\
$P(0)$ gilt: Wegen 1)\\
\\
$(P(0) \wedge (P(0) \Rightarrow P(1)) \Rightarrow P(1)$\\
\tab wegen 2) mit $n=0$\\
\\
$(P(1) \wedge (P(1) \Rightarrow P(2)) \Rightarrow P(2)$\\
\tab wegen 2) mit $n=1$\\
\\
Damit gilt $P(n)$ fuer alle $n:\mathbb{N}$\\
\\
\\
\uline{Unser Beispiel}\\
\\
1) \uline{Induktionsanfang} (Base case)\\
zu zeigen: $P(0)$\\
$0^3 + 5*n = 0$(Zeuge) $*\:6$\\
\\
2) \uline{Induktionsschritt} (inductive step)\\
zu zeigen: $P(n) \Rightarrow P(n+1)$ fuer alle $n:\mathbb{N}$\\
\\
Sei $n$ eine \uline{beliebige} natuerliche Zahl.\\
\\
\uline{Annahme}: Es gaelte $P(n) : n^3 + 5 * n$, dass heisst $n^3 + 5 * n = 6 * r$, ist vielfaches von $6$\\
zu zeigen: (unter diese Annahme) $P(n+1): (n+1)^3 + 5 * (n+1)$ ist vielfaches von $6$\\
das heisst: $(n+1)^3 + 5 * (n+1) = 6 * s$ mit $s: \mathbb{Z}$\\
\\
$(n+1)^3 + 5 * (n+1)$\\
\tab $<$Arith$>$ \\
$= (n^3 + 3*n^2 + 3*n +1) + (5*n + 5)$\\
\tab $<$Arith + Kaninchen$>$ \\
$= (n^3 + 5*n) + (3*n^2 + 3*n + 6)$\\
\tab $<$Annahme$>$\\
$= 6*r + 3*n^2 + 3*n + 6$\\
\tab $<$Arith + Kaninchen$>$\\
$= 6*r + 3*n*(n+1) + 6$\\
\tab $<$n*(n+1) ist gerade$>$\\
$= 6*r + 3*(2*t) + 6$\\
\tab $<$Arith$>$\\
$= 6*(r+t+1)$ (Zeuge) \checkmark \\
\\








%\subsection{This is a subsection}

%
% TODO: This is a comment
% COMMENT
%


\end{document}